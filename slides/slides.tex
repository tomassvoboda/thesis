\documentclass{beamer}
\usepackage[utf8]{inputenc}
\usepackage[czech]{babel}
\usepackage{palatino}

\usepackage{tikz}
\usetikzlibrary{automata}
\usetikzlibrary{positioning}
\usetikzlibrary{arrows}
\tikzset{my_automaton/.style={->,>=latex,semithick,bend angle=20}}
\tikzset{my_state/.style={circle, very thick, minimum size=0.4cm}}
\tikzset{my_named_state/.style={circle, thick, minimum size=0.7cm, inner sep=0.05cm}}
\tikzset{my_initial_state/.style={initial by arrow, initial text=}}
\tikzset{my_accepting_state/.style={accepting by double, thick}}

\usetheme{Rochester}
\title[Hvězdná výška]{Hvězdná výška}
\author[Tomáš Svoboda]{Tomáš Svoboda}
\institute{Katedra algebry}
\date{2017}

\usefonttheme{professionalfonts}

%%% The field of all real, natural, and whole numbers
\newcommand{\R}{\mathbb{R}}
\newcommand{\N}{\mathbb{N}}
\newcommand{\Z}{\mathbb{Z}}

% Command for power set
\newcommand{\ps}[1]{\mathbb{P}(#1)}
% Command for rational expression
\newcommand{\re}[1]{{\mathsf{#1}}}
% Commands for star height and generalised star height
\newcommand{\h}[1]{{\mathsf{h}[#1]}}
\newcommand{\gh}[1]{{\mathsf{gs h}[#1]}}
% Command for automaton
\newcommand{\aut}[1]{{\mathcal{#1}}}
% Command for transition
\newcommand{\tr}[2][]{{\; \xrightarrow[#1]{\: \; #2 \: \;} \;}}


\begin{document}
\setbeamertemplate{navigation symbols}{}
\setbeamerfont{title page}{family=\rmfamily}

\begin{frame}
\titlepage
\end{frame}

\begin{frame}{Definice}
      \begin{exampleblock}{Abeceda}
            \centering
            $\{ a, b, c, d \}, \: \{a, b\}$
      \end{exampleblock}

      \pause

      \begin{exampleblock}{Slovo}
            \centering
            $aaa, \: abcb, \: d, \: 1$
      \end{exampleblock}

      \pause

      \begin{exampleblock}{Jazyk}
            \centering
            $\{ aaa, \: abcb, \: d \}, \: \{ 1, \: ab, \: abab, \: ababab, \: \dots \}$
      \end{exampleblock}
\end{frame}

\begin{frame}{Definice}
      \begin{exampleblock}{Racionální výraz}
            \vspace{-5mm}%
            \begin{align*}
                  &L[\re{0}] = \emptyset \\
                  &L[\re{1}] = \{1\} \\
                  &L[a] = \{a\} \quad \text{pro každé $a$ v $A$} \\
                  \\
                  &L[a(a+b) + b] = \{aa, \: ab, \: b\} \\
                  \\
                  &L[{(ab)}^*] = \{\re{1}, \: ab, \: abab , \: \dots \}
            \end{align*}
      \end{exampleblock}
\end{frame}

\begin{frame}{Hvězdná výška}
      \begin{exampleblock}{Příklad}
            \vspace{-5mm}%
            \begin{align*}
                  \h{a(a+b) + b} &= 0 \\
                  \h{(a+b)^*} &= 1 \\
                  \h{a^*(ba^*)^*} &= 2 \\
                  \h{a^* + b(ab^*a + ba^*b)^*ba^*} &= 2
            \end{align*}
      \end{exampleblock}

      ~

      \only<2>
      {
            \begin{exampleblock}{Příklad}
                  \vspace{-5mm}%
                  \begin{align*}
                        \h{a^* + b(ab^{\color{red}\boldsymbol{*}}a + ba^*b)^{\color{red}\boldsymbol{*}}ba^*} &= 2 \\
                        \h{a^* + b(ab^*a + ba^{\color{red}\boldsymbol{*}}b)^{\color{red}\boldsymbol{*}}ba^*} &= 2
                  \end{align*}
            \end{exampleblock}
      }
      \only<3>
      {
            \begin{block}{Hvězdná výška jazyka}
                  Minimum hvězdných výšek všech výrazů, které popisují daný jazyk.
            \end{block}
      }
\end{frame}

\begin{frame}{Kontext}
      \begin{block}{Otázka}
            Existuje jazyk s hvězdnou výškou $n$, pro každé přirozené $n$?
      \end{block}

      ~

      \begin{itemize}
            \item Výsledek z roku 1966
            \item Tato práce přispívá detailnější důkazy
      \end{itemize}
\end{frame}

\begin{frame}{Automat}
      \begin{exampleblock}{Automat~$\aut{A}$}
            \centering
            \begin{tikzpicture}[my_automaton]
    \tikzstyle{every state}=[my_named_state]
    \tikzstyle{initial}=[my_initial_state]
    \tikzstyle{accepting}=[my_accepting_state]
    %\draw[help lines] (-1.8,1.8) grid (1.8,-1.8);
    \node[state, initial] (0) at (0,0) {$p$};
    \node[state] (1) at (2,2) {$q$};
    \node[state, accepting] (2) at (2,0) {$r$};
    \node[state, initial, initial above, accepting] (3) at (4,0) {$s$};

    \path
        (0) edge [] node [above] {$a$} (2)
        (1) edge [] node [above left] {$aa+b$} (0)
        (2) edge [] node [right] {$bb$} (1)
        (2) edge [bend left] node [above] {$a$} (3)
        (3) edge [loop right] node {$ab$} ()
        (3) edge [bend left] node [near end, below] {$b$} (2)
    ;
\end{tikzpicture}
      \end{exampleblock}

      ~
      \pause

      \begin{block}{Jazyk příjímaný automatem}
            Cesta $p \tr{a} r \tr{bb} q \tr{aa+b} p \tr{a} q$. \\
            Konkatenace: $a \cdot bb \cdot (aa+b) \cdot a = abb(aa+b)a$. \\
            Sjednocení jazyků: $L[abb(aa+b)a]$.
      \end{block}

\end{frame}

\begin{frame}{Reprezentace jazyka}
      \begin{block}{Kleeneova věta}
            Třída jazyků popsatelných racionálním výrazem je stejná jako třída jazyků příjímaných automatem.
      \end{block}
\end{frame}

\begin{frame}{Prstencový automat}
      \begin{exampleblock}{Prstencový automat~$\aut{R}(8)$}
            \begin{center}
                  \begin{tikzpicture}[my_automaton]
    \tikzstyle{every state}=[my_state]
    \tikzstyle{initial}=[my_initial_state]
    \tikzstyle{accepting}=[my_accepting_state]
    %\draw[help lines] (-1.5,1.5) grid (1.5,-1.5);
    \node[state, initial, accepting] (0) at (180:1.5) {};
    \node[state] (1) at (135:1.5) {};
    \node[state] (2) at (90:1.5) {};
    \node[state] (3) at (45:1.5) {};
    \node[state] (4) at (0:1.5) {};
    \node[state] (5) at (-45:1.5) {};
    \node[state] (6) at (-90:1.5) {};
    \node[state] (7) at (-135:1.5) {};

    \path
        (0) edge [above, bend left] node [near start, above left=-1mm] {$a$} (1)
        (1) edge [above, bend left] node [right] {$b$} (0)
        (1) edge [above, bend left] node [above] {$a$} (2)
        (2) edge [above, bend left] node [below right=-1mm] {$b$} (1)
        (2) edge [above, bend left] node [above right=-1mm] {$a$} (3)
        (3) edge [above, bend left] node [near start, below left=-1mm] {$b$} (2)
        (3) edge [above, bend left] node [right] {$a$} (4)
        (4) edge [above, bend left] node [left] {$b$} (3)
        (4) edge [above, bend left] node [right] {$a$} (5)
        (5) edge [above, bend left] node [above left=-1mm] {$b$} (4)
        (5) edge [above, bend left] node [below] {$a$} (6)
        (6) edge [above, bend left] node [above] {$b$} (5)
        (6) edge [above, bend left] node [below] {$a$} (7)
        (7) edge [above, bend left] node [above] {$b$} (6)
        (7) edge [above, bend left] node [left] {$a$} (0)
        (0) edge [above, bend left] node [above right=-1mm] {$b$} (7)
    ;
\end{tikzpicture}
            \end{center}
      \end{exampleblock}
\end{frame}

\begin{frame}{Jazyk $W_q$}
      \begin{block}{Definice}
            $W_q$ je jazyk na ${\{a, b\}}$ obsahující všechna slova jejichž počet písmen $a$ je kongruentní počtu písmen $b$ modulo $2^q$.
      \end{block}

      ~
      \pause

      \begin{block}{Lemma}
            Prstencový automat~$\aut{R}(2^q)$ rozpoznává jazyk $W_q$.
      \end{block}
\end{frame}

\begin{frame}{Hvězdná výška $W_q$}
      \begin{block}{Lemma}
            Jazyk $W_q$ je označován racionálním výrazem hvězdné výšky~$q$.
      \end{block}

      ~
      \pause

      \begin{block}{Algoritmus odstraňování stavů}
            Nechť $n$ stavový automat má alespoň jeden stav, který není počáteční, ani koncový. Pak existuje automat s $n-1$ stavy příjímající stejný jazyk.
      \end{block}
\end{frame}

\begin{frame}{Redukovaný automat}
      \begin{exampleblock}{Redukovaný automat~$\aut{R}(2^q)$}
            \begin{center}
                  \begin{tikzpicture}[my_automaton]
    \tikzstyle{every state}=[my_named_state]
    \tikzstyle{initial}=[my_initial_state]
    \tikzstyle{accepting}=[my_accepting_state]
    %\draw[help lines] (-1.8,1.8) grid (1.8,-1.8);
    \node[state, initial, accepting] (0) at (0,0) {$0$};

    \path
        (0) edge [loop right] node {$\re{E}$} ()
    ;
\end{tikzpicture}
            \end{center}
      \end{exampleblock}
\end{frame}

\begin{frame}{Odpovědi}
      \pause

      \begin{block}{Lemma 5}
            Část, že $L(\aut{B}) \subseteq L(\aut{A})$ pro $n-1$ stavový $\aut{B}$ a $n$ stavový $\aut{A}$.
      \end{block}

      \begin{block}{Důkaz}
            Nechť $i \tr{\re{E}} t$ je úspěšný výpočet v $\aut{B}$. \\
            \quad $i \tr{\re{E}} t$ je úspěšný v $\aut{A}$ se stejnou faktorizací jako v $\aut{B}$. \\
            \quad Existuje $\re{E} = \re{F}{\re{G}}^*\re{H}$, že $i \tr{\re{F}} q \tr{\re{G}} \dotsm \tr{\re{G}} q \tr{\re{H}} t$. \\
            Pak je $i \tr{\re{F}} q \tr{\re{G}} \dotsm \tr{\re{G}} q \tr{\re{H}} t$ úspěšný v $\aut{A}$ \\
      \end{block}
\end{frame}

\begin{frame}{Odpovědi}
      \begin{block}{Lemma 7}
            Prstencový automat~$\aut{R}(2^q)$ rozpoznává jazyk $W_q$.
      \end{block}

      \begin{block}{Důkaz}
            Nechť $w$ je slovo na $\{a,b\}$ délky $n$. \\
            Ukážeme, že pro $k \equiv |w|_a - |w|_b \pmod{2^q}$ je $0 \tr{w} k$ v $\aut{R}(2^q)$.\\
            Platí pro $n = 0$, protože $w = 1$. \\
            Nechť $u$ je prefix $w$ délky $n-1$. Pak $w$ je buď $ua$, nebo $ub$. \\
            Nechť $l \equiv |u|_a - |u|_b \pmod{2^q}$. \\
            \quad Pro $w = ua$, $l \equiv k-1 \pmod{2^q}$ \\
            \quad Pro $w = ub$, $l \equiv k+1 \pmod{2^q}$ \\
            $0 \tr{u} l, \; k-1 \tr{a} k, \; k+1 \tr{b} k \;$ jsou v $\aut{R}(2^q)$. \\
            \quad Pro $w = ua$, $0 \tr{u} k-1 \tr{a} k$ je v $\aut{R}(2^q)$. \\
            \quad Pro $w = ub$, $0 \tr{u} k+1 \tr{b} k$ je v $\aut{R}(2^q)$. \\
      \end{block}
\end{frame}

\end{document}
