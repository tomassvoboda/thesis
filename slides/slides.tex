\documentclass{beamer}
\usepackage[utf8]{inputenc}
\usepackage[czech]{babel}
\usepackage{palatino}
\usetheme{Warsaw}
\title[Medvědův průvodce bakalářkou]{Medvědův průvodce bakalářkou}
\author[Martin Mareš]{Martin Mareš\\\texttt{mj@ucw.cz}}
\institute{Katedra Aplikované Matematiky\\MFF UK Praha}
\date{2017}
\begin{document}
\setbeamertemplate{navigation symbols}{}
\setbeamerfont{title page}{family=\rmfamily}

\begin{frame}
\titlepage
\end{frame}

\begin{frame}{Odevzdávání práce}

Formality určují Směrnice děkana 8/2010, 2/2015 a 7/2016
a Opatření rektora 13/2017.

~

Papírová verze:

\begin{itemize}
\item 3 výtisky: vedoucí, oponent, knihovna
\item Oponentský výtisk se po obhajobě vrací autorovi
\item Požadována pevná vazba
\item Odevzdává se paní Košnářové z~KTIML
\end{itemize}

~

Elektronická verze:

\begin{itemize}
\item PDF/A (level 1a nebo 2u) + přílohy (omezení formátů!)
\item Odevzdává se do SISu
\item Pokud jsou přílohy příliš velké, odevzdává se též
      na DVD vlepeném do desek papírové verze
\end{itemize}

\end{frame}

\begin{frame}{Úprava práce}

Povinná úprava práce:

\begin{itemize}
\item Směrnice děkana 1/2015 a 5/2016
\item Formát papíru a vazba
\item Formality: desky, titulní strana, informační strana, \dots
\end{itemize}

Doporučená úprava práce:

\begin{itemize}
\item \url{http://www.mff.cuni.cz/studium/bcmgr/prace/}
\item Upřesňuje formát: okraje, velikost písma
\item Jednostranný / oboustranný tisk (u delších prací)
\item Obsah, seznam zkratek, obrázků a tabulek, \dots
\item {\it Chcete-li něco dělat jinak, poraďte se s vedoucím.}
\end{itemize}

K~dispozici je šablona pro \LaTeX:

\begin{itemize}
\item Pracovní verze na \url{http://mj.ucw.cz/vyuka/bc/}
\end{itemize}

\end{frame}

\begin{frame}{Jazykové zákoutí}

\begin{itemize}
\item Jakým jazykem: CS / SK / EN; jiný se souhlasem děkana
   \begin{itemize}
   \item Bakalářka by neměla být vaším prvním anglickým textem
   \end{itemize}
\item Jazyková kultura:
   \begin{itemize}
   \item Spisovný a přesný jazyk (žádná barbarština)
   \item Terminologie: vyhýbejte se czengličtině i hyperkorektnosti
   \item Slovosled (HTTP protokol, kompaktní CD disk)
   \item Pravopys a syntaxe
   \end{itemize}
\item Obvykle píše{\bf me} v~1. osobě množného čísla.
\item Vyhýbejte se složitým větám, potažmo souvětím (a~spoustě
   závorek (že?)), jakož i dlouhatánským odstavcům.
\item Nepište detektivky. Hned řekněte, že vrah je zahradník.
\item Zamyslete se nad pořadím. Dopředné odkazy radši ne.
\item Pište jednotně.
\end{itemize}

\end{frame}

\begin{frame}{Jazykové zákoutí 2}

\begin{itemize}
\item Rozmyslete si, pro koho píšete:
   \begin{itemize}
   \item Pro kolegu matfyzáka (oponent, váš spolužák, \dots)
   \item Čtenář má obecné informatické vzdělání (není třeba definovat graf),
         ale není expertem přes váš obor.
   \end{itemize}
\item Stručnost je sestra talentu.
\item Zkuste si text přečíst nahlas.
\item Testujte na lidech.
\item Vnímejte, jak píši jiní. Ale s~rozmyslem.
\end{itemize}

\end{frame}

\begin{frame}{Typografické zákoutí}

\begin{itemize}
\item Použijte program určený k~sazbě textu (\TeX, Scribus, \dots),
   Word v~nouzi poslouží, ale výstup obvykle vypadá hrozně.
\item Zvolte si typ písma pro {\bf definice,} {\tt identifikátory,} {\it zvýraznění\/} \dots{} a
   dodržujte ho v~celém textu.
\item S~typy písma a barvami to nepřehánějte.
\item Různé typy pomlček: Fordův-Fulkersonův, 100--200 bitů, případně -- jako tady -- kolem vsuvky.
\item Desetinné tečky {\it xor\/} čárky.
\item \clqq~české uvozovky \crqq
\item Nedělitelné mezery: u{\tt\char32}předložek, vrchol{\tt\char32}$v$, a{\tt\char32}proto.
\item Obrázky nejlépe vektorové nebo bitmapy ve~vysokém rozlišení. JPEG jen na fotky.
\end{itemize}

\end{frame}

\begin{frame}{Formát práce: PDF/A}

\begin{itemize}
\item PDF/A je profil formátu PDF určený pro archivaci
\item Různé varianty: na UK povoleno PDF/A-1a nebo PDF/A-2u
\item Nesmí obsahovat:
	\begin{itemize}
	\item odkazy na externí fonty (včetně systémových)
	\item audio, video, skripty, obskurnosti
	\item transparentní vrstvy (povoleny ve 2u)
	\end{itemize}
\item Musí obsahovat:
	\begin{itemize}
	\item deklarace barevných prostorů
	\item meta-data ve formátu XMP
	\end{itemize}
\item Ve všelijakých officech alespoň formálně OK
\item V~\LaTeX{}u použijte balíček {\bf pdfx} (viz šablona)
\item Pozor na vložené obrázky (záchrana GhostScriptem)
\item Kontrola pomocí VeraPDF (\url{http://verapdf.org/})
\end{itemize}

\end{frame}

\begin{frame}{Formát práce: Přílohy}

Povolené formáty příloh jsou tyto:

\begin{itemize}
\item Text: PDF/A
\item Obrázky: JPEG, PDF/A
\item Zvuk: WAV, MP3
\item Video: MPEG2, MPEG4
\item Tabulky: CSV, XML, TXT
\end{itemize}

Ostatní po schválení fakultním koordinátorem.

\end{frame}

\begin{frame}{Citace: Proč}

\begin{block}{Příklad}
Největší problém citátů na internetu je ten,
že nevíte, jestli jsou pravé.

\rightline{\it Josef Čapek\qquad}
\end{block}

\end{frame}

\begin{frame}{Citace: Co a jak}

\begin{itemize}
\item Za~cizí myšlenky se nestyďte, ale správně je citujte.
   \begin{itemize}
   \item Použité věty, algoritmy, důkazy
   \item Celá pasáž cizího textu -- měla by být jasně ohraničena
   \end{itemize}
\item Odkazy: \uv{$\,$O~{\it temnu\/} píší jak Balcar se Štěpánkem [1], tak Jirásek [2].}
\item {\bf Špatně:} \uv{$\,$Jak najdeme v~[3]}
\item Číslování: abecedně nebo podle pořadí odkazování
\item Alternativně: [BŠ2005]~~~(hodí se, máte-li odkazů hodně)
\end{itemize}

\begin{itemize}
\item Vhodné zdroje (kriteria: kredit, spolehlivost, dostupnost):
   \begin{itemize}
   \item Původní publikace (typicky článek)
   \item Monografie
   \item Učebnice
   \item Lze i osobní komunikace :)
   \item Raději ne encyklopedie
   \item Rozhodně ne náhodné webové stránky
   \end{itemize}
\end{itemize}

\end{frame}

\begin{frame}{Citace: Formát bibliografie}

\begin{itemize}
\item Je doporučeno držet se ISO 690.
\item Norma je ovšem obtížně dostupná (kdo ji četl?).
\item Praxe: různé výtahy, třeba na \url{http://citace.com/}
\item Pozor na rozpory normy se zvyklostmi oboru.
\end{itemize}

\vfill

{\sc Balcar}, Bohuslav a {\sc Štěpánek}, Petr. {\it Teorie množin.\/}
2.~vydání. Praha: Academia, 2005. ISBN 80-200-0470-X.

\medskip

{\sc Marný}, Tomáš. Tužkožrout obecný: tajuplný obyvatel akademické půdy.
In: {\it Časopis pro pěstování strašidel.} \\
2010, č.~5, s.~60--69. ISSN 1234-5678.

\medskip

{\sc Munroe,} Randall. Random Number.
In: {\it XKCD: A~webcomic of romance, sarcasm, math and language\/} [online].
2010 \hfil\break
[cit. 2011-10-04]. Dostupné~z: \url{http://xkcd.com/221/}
\end{frame}

\begin{frame}{Citace: Nástroje}

\begin{itemize}
\item Formátování bibliografie:
   \begin{itemize}
   \item Bib\TeX{} -- poněkud obskurní, ale účinný
   \item Bib\LaTeX{} -- novější, ohebnější
   \item Pybibliographer -- pokud nepoužíváte \TeX
   \end{itemize}
\bigskip
\item Sběr bibliografie:
   \begin{itemize}
   \item citeulike.org
   \item Google Scholar
   \item Citeseer
   \item \dots{} každopádně nutno upravovat ručně
   \end{itemize}
\end{itemize}

\end{frame}

\end{document}
