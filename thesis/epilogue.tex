\chapter*{Conclusion}
\addcontentsline{toc}{chapter}{Conclusion}

This thesis proves, in greater detail than the original works, that there is an~infinite hierarchy of star heights of languages over binary alphabet.

In the first chapter, we defined an~unusual modification of classical automata with transitions labelled with rational expressions. This let us throughout the thesis consider only automata with at most one transition between each pair of states. We devised a~precise definition of a~\emph{ring automaton} and proved Lemma~\ref*{lm:R_n_computation_existence} about the existence of computations in ring automata. For the state removal algorithm we compared the successful computations in $n$ and $n-1$ state automata to prove their equivalence.

In the second chapter, the aim of this thesis was to take the proof of existence of languages with arbitrarily high star height, as formulated by Sakarovitch~\cite{Sakarovitch09}, and in detail prove the parts that were only outlined. In the Section~\ref*{section:automaton_recognising_W_q} we used our Lemma~\ref*{lm:R_n_computation_existence} to show that language $W_q$ is recognised by the ring automaton with $2^q$ states. In the Section~\ref*{section:rational_expression_denoting_W_q} we showed that, with a~particular order of states chosen for state removal algorithm, language $W_q$ is denoted by a~rational expression of star height $q$. In the Section~\ref*{section:witness_words} witness words are defined. In Lemma~\ref*{lm:witness_words_equality} we have shown that each of them has the same number of both letters of the binary alphabet over which they are defined. This equality is used to simplify the proof of Lemma~\ref*{lm:witness_words_inequalities}, where we show how much more letters of one type can be in either prefix, or suffix of the witness words. This inequality is used in the proof of \autoref*{thm:main}.