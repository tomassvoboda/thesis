\chapter{Introduction}

In the first section we state same definitions that are considered standard and lay down notation that is used later in the work. In the second section we provide some additional definitions and observations that are more specific to the presented problem.

\section{Basic definitions \& notation}

\begin{defn} \X{List of definitions and notation that need to be introduced:}\\
    \X{Word, factors, $|\cdot|_a$, ``word is a factor of language''}\\
    \X{Automaton, finite automaton, automaton equivalence, generalised automaton}
\end{defn}

\begin{defn}
    \X{Expression, expression equivalence, star height (of expression, of language)}
\end{defn}

\begin{lemma}\label{lm:star_height_distributivity}
    Every language $L$ over $A$ with star height~$h$ is denoted by a finite sum of expressions:
    \[
        L = \sum_{j=1}^n \mathsf{E}_j,
    \]
    where each $\mathsf{E}_j$ if product of the form:
    \[
        \mathsf{E}_j = u_0{(\mathsf{F}_1)}^*u_1 \dotsm u_{m-1}{(\mathsf{F}_m)}^*u_m,
    \]
    where each $u_k, 0 \leq k \leq m$, is a word in $A^*$ and each $\mathsf{E}_j, 1 \leq k \leq m$, is an expression over $A$ of star height less then or equal to $h-1$.
\end{lemma}

\begin{proof}
    By induction on $h$. For language of star height 1 the fact that sum distributes over product gives us the result immediately. \X{Finish the proof.}
\end{proof}

\begin{lemma}\label{lm:block_star_lemma}
    Let $L$ be a recognisable language over $A$. There exists $N \in \N$ that for every word $f$ in $L$ and every factorisation of the form
    \[
        f = u v_1 v_2 \dotsm v_N w \; , \quad \text{where every $v_i$ is a non-empty word.}
    \]
    Then there is a pair $(j,k)$ of indices, $0 \leq j < k \leq N$, that
    \[
        u v_1 v_2 \dotsm v_j {(v_{j+1} \dotsm v_k)}^* v_{k+1} \dotsm v_N w \subseteq L.
    \]
\end{lemma}

\subsection{State removal algorithm}

State removal algorithm takes provided finite automaton and derives a~rational expression that denotes the same language as is accepted by the original automaton. Let $\mathcal{A}~=~\langle Q, A, E, I, T \rangle$ be the provided automaton. Next, we construct an~automaton $\mathcal{B}$, from $\mathcal{A}$, by adding two states $i$ and $t$, distinct and not belonging to $Q$, a spontaneous transition from $i$ to each initial state of $\mathcal{A}$, and a spontaneous transition from each final state of $\mathcal{A}$ to $t$. We consider the state $i$ to be the only initial state, and $t$ the only final state, of automaton $\mathcal{B}$. Clearly $\mathcal{A}$ and $\mathcal{B}$ are equivalent and since $\mathcal{A}$ is finite, $\mathcal{B}$ is also finite.

Next part of the algorithm has as many steps as there are states in $Q$. In each step we remove a state in $Q$ from $\mathcal{B}$ and modify the transitions in a way that the derived automaton is equivalent to that of the previous step. Let $q$ be a~particular state in $Q$, let $p_1, \dotsc , p_k$ be states of $\mathcal{B}$, $p_j \neq q$, that are the sources of transitions which have $q$ as a destination, and $\re{E}_1, \dotsc , \re{E}_k$ the labels of these transitions. And let $r_1, \dotsc , r_l$ be states of $\mathcal{B}$, $r_j \neq q$, that are the destinations of transitions whose source is $q$, and $\re{G}_1, \dotsc , \re{G}_l$ the labels of these transitions. Note that some states $p_j$ can be the same as states $r_m$. If there is a transition with source and destination $q$, we will denote it by $\re{F}$. Otherwise we set $\re{F} = 0$ and therefore we have $\re{F}^* = 1$.

Let $\mathcal{B}'$ be an automaton obtained from $\mathcal{B}$ by first removing the state $q$ and all of the following transitions described above
\[
    (p_1, \re{E}_1, q), \dotsc, (p_k, \re{E}_k, q), (q, \re{G}_1, r_1), \dotsc, (q, \re{G}_l, r_l), (q, \re{F}, q),
\]
and second by adding, for each pair of states $(p_h, r_j)$, $1 \leq h \leq k$ and $1 \leq j \leq l$, the transition $(p_h, \re{E}_h \re{F}^* \re{G}_j, r_j)$.

\section{\X{Additional definitions}}

\begin{defn}
    Let $A = {\{a,b\}}$. An automaton $\langle \Z_n, A, E, 0, 0 \rangle$ is called \emph{ring automaton~${\mathcal{R}(n)}$}, if for each state $z \in \Z_n$ there are exactly two transitions with $z$ as a~source and they are $(z, a, z+1)$ and $(z, b, z-1)$.
\end{defn}

\begin{example}
    \autoref*{fig:automaton_R8} shows ring automaton~${\mathcal{R}(8)}$.

    \begin{figure}[h]
        \centering
        \begin{tikzpicture}[my_automaton]
    \tikzstyle{every state}=[my_state]
    \tikzstyle{initial}=[my_initial_state]
    \tikzstyle{accepting}=[my_accepting_state]
    %\draw[help lines] (-1.5,1.5) grid (1.5,-1.5);
    \node[state, initial, accepting] (0) at (180:1.5) {};
    \node[state] (1) at (135:1.5) {};
    \node[state] (2) at (90:1.5) {};
    \node[state] (3) at (45:1.5) {};
    \node[state] (4) at (0:1.5) {};
    \node[state] (5) at (-45:1.5) {};
    \node[state] (6) at (-90:1.5) {};
    \node[state] (7) at (-135:1.5) {};

    \path
        (0) edge [above, bend left] node [near start, above left=-1mm] {$a$} (1)
        (1) edge [above, bend left] node [right] {$b$} (0)
        (1) edge [above, bend left] node [above] {$a$} (2)
        (2) edge [above, bend left] node [below right=-1mm] {$b$} (1)
        (2) edge [above, bend left] node [above right=-1mm] {$a$} (3)
        (3) edge [above, bend left] node [near start, below left=-1mm] {$b$} (2)
        (3) edge [above, bend left] node [right] {$a$} (4)
        (4) edge [above, bend left] node [left] {$b$} (3)
        (4) edge [above, bend left] node [right] {$a$} (5)
        (5) edge [above, bend left] node [above left=-1mm] {$b$} (4)
        (5) edge [above, bend left] node [below] {$a$} (6)
        (6) edge [above, bend left] node [above] {$b$} (5)
        (6) edge [above, bend left] node [below] {$a$} (7)
        (7) edge [above, bend left] node [above] {$b$} (6)
        (7) edge [above, bend left] node [left] {$a$} (0)
        (0) edge [above, bend left] node [above right=-1mm] {$b$} (7)
    ;
\end{tikzpicture}
        \caption{Automaton~${\mathcal{R}(8)}$}\label{fig:automaton_R8}
    \end{figure}
\end{example}