\chapter{Introduction}

In the first section we state same definitions that are considered and lay down notation that is used in the work. In the second section we provide some additional definitions and observations that are more specific to the presented problem.

\section{Basic definitions \& notation}

\begin{defn} \X{List of definitions and notation that need to be introduced:}\\
    \X{Word, factors, $|\cdot|_a$}\\
    \X{Automaton}\\
    \X{Expression, expression equivalence, star height (of expression, of language)}
\end{defn}

\begin{defn} \X{State removal algorithm}\\
\end{defn}

\begin{lemma} \X{Distributivity od sum over product}\\
\end{lemma}

\section{\X{Additional definitions}}

\begin{defn}
    An automaton $\mathcal{A} = \langle Q, {\{a,b\}}^*, E, \{p\}, \{p\} \rangle$ is called \emph{ring automaton~${\mathcal{R}(n)}$}, if there is a bijection $\varphi: \Z_n \to Q$, that
    \[
        \forall \, z \in Z_n \; \exists \: (\varphi(z),a,\varphi(z+1)), (\varphi(z),b,\varphi(z-1)) \in E.
    \]
    \X{$\; \uparrow$ Not fully flashed out yet}
\end{defn}

\begin{example}
    \autoref*{fig:automaton_R8} shows ring automaton~${\mathcal{R}(8)}$.
\end{example}

\begin{figure}[h]
    \centering
    \begin{tikzpicture}[my_automaton]
    \tikzstyle{every state}=[my_state]
    \tikzstyle{initial}=[my_initial_state]
    \tikzstyle{accepting}=[my_accepting_state]
    %\draw[help lines] (-1.5,1.5) grid (1.5,-1.5);
    \node[state, initial, accepting] (0) at (180:1.5) {};
    \node[state] (1) at (135:1.5) {};
    \node[state] (2) at (90:1.5) {};
    \node[state] (3) at (45:1.5) {};
    \node[state] (4) at (0:1.5) {};
    \node[state] (5) at (-45:1.5) {};
    \node[state] (6) at (-90:1.5) {};
    \node[state] (7) at (-135:1.5) {};

    \path
        (0) edge [bend left] node [near start, above left=-1mm] {$a$} (1)
        (1) edge [bend left] node [right] {$b$} (0)
        (1) edge [bend left] node [above] {$a$} (2)
        (2) edge [bend left] node [below right=-1mm] {$b$} (1)
        (2) edge [bend left] node [above right=-1mm] {$a$} (3)
        (3) edge [bend left] node [near start, below left=-1mm] {$b$} (2)
        (3) edge [bend left] node [right] {$a$} (4)
        (4) edge [bend left] node [left] {$b$} (3)
        (4) edge [bend left] node [right] {$a$} (5)
        (5) edge [bend left] node [above left=-1mm] {$b$} (4)
        (5) edge [bend left] node [below] {$a$} (6)
        (6) edge [bend left] node [above] {$b$} (5)
        (6) edge [bend left] node [below] {$a$} (7)
        (7) edge [bend left] node [above] {$b$} (6)
        (7) edge [bend left] node [left] {$a$} (0)
        (0) edge [bend left] node [above right=-1mm] {$b$} (7)
    ;
\end{tikzpicture}
    \caption{Automaton~${\mathcal{R}(8)}$}\label{fig:automaton_R8}
\end{figure}