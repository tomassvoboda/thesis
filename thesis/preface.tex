\chapter*{Introduction}
\addcontentsline{toc}{chapter}{Introduction}

Kleene~\cite{Kleene56}, in result known as Kleene’s Theorem, shows that automata and expressions correspond to each other and characterise the same class of languages. Eggan~\cite{Eggan63} refines this result by defining a measure of complexity for both of them: loop complexity for automata and star height for expressions, and by showing they correspond to each other by characterising the same classes of languages. In this thesis we limit ourselves to the presentation of star height, and to the proof, due to Dejean and Schützenberger~\cite{DejeanSchutzenberger66}, which states that the star height hierarchy is infinite. They show that for any integer $k \leq 0$, there exists a~rational language of star height $k$ over two letter alphabet. We end by touching on another notion, the generalised star height, which may divide the family of rational languages into only two parts.

The determination of the star height of a~language turns out to be one of the most difficult problems in automata theory. McNaughton~\cite{McNaughton67} presented first notable result, an~algorithm for finding the star height of certain family of languages, so called \emph{pure-group languages}. Hashiguchi first~\cite{Hashiguchi1982} provided an~algorithm for deciding whether or not an~arbitrary rational language is of star height one and then~\cite{Hashiguchi1988}, after six years, an~algorithm to determine the star height of any rational language. The algorithm for the general case was not practical, being of non-\textsc{ELEMENTARY} complexity class. Kirsten~\cite{Kirsten05} devised a~more efficient algorithm than Hashiguchi's, decidable in $2^{2^{\aut{O}(n)}}$ space.