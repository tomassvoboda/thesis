\chapter{Eggan's question}

\cite{Eggan63} asks whether there are languages with arbitrarily large star height. In this section we present proof first due to~\cite{DejeanSchutzenberger66} and recently formulated in~\cite{Sakarovitch09}.

Throughout this chapter we use language $W_q = {\{f \mid |f|_a \equiv |f|_b \bmod 2^q \}}$.

\begin{thm}\label{thm:main}
    The language $W_q$ has star height~$q$.
\end{thm}

For the~proof we need to find an~expression of star height~$q$ denoting~$W_q$. That is done in~Lemma~\ref*{lm:expression_existence}. Then it remains to show that $W_q$ has the star height of at~least~$q$.

\begin{lemma}\label{lm:expression_existence}
    For each $q \in \N$ there is an expression of star height~$q$, that denotes~$W_q$.
\end{lemma}

\begin{proof}
    $W_q$ is recognised by an automaton~${\mathcal{R}(2^q)}$. By following a specific order~$\omega$ in the elimination method on $W_q$ we get an expression of a~star height $q$. First we set expressions
    \begin{alignat*}{5}
        X_1 = a^2 , \quad Y_1 = b^2, \quad &\text{ and } \quad Z_1 = ab + ba.
    \intertext{Then for every integer $n$ we have}
        X_{n+1} = X_n Z_n^* X_n , \quad Y_{n+1} = Y_n Z_n^* Y_n, \quad &\text{ and } \quad Z_{n+1} = Z_n + X_n Z_n^* Y_n + Y_n Z_n^* X_n.
    \end{alignat*}
    It follows that
    \[
        W_q = {(X_q + Y_q + Z_q)}^*.
    \]
    Since $X_n , Y_n , \text{ and } Z_n$ have star height~$n~-~1$, we have got an~expression denoting~$W_q$ of star height~$q$.
\end{proof}

\begin{example}
    Let us have language~$W_3$ recognised by an automaton~${\mathcal{R}(8)}$. In~\autoref*{fig:automaton_R8_state_removal_steps} we show how the~state removal algorithm finds the~expression of star height~$3$ that denotes~$W_3$. For the expression at the end of the algorithm we can see following equivalence:
    \[
        {(Z_2 + {(X_2 + Y_2)}Z_2^*{(X_2 + Y_2)})}^* = {(X_3 + Y_3 + Z_3)}^* = W_3.
    \]
\end{example}

\begin{figure}%
    \centering
    \subfloat[][]{%
        \label{fig:automaton_R8_state_removal_steps-a}%
        \input{./figures/automaton_R8_removed1}}%
    \hspace{70pt}%
    \subfloat[][]{%
        \label{fig:automaton_R8_state_removal_steps-b}%
        \begin{tikzpicture}[my_automaton]
    \tikzstyle{every state}=[my_state]
    \tikzstyle{initial}=[my_initial_state]
    \tikzstyle{accepting}=[my_accepting_state]
    %\draw[help lines] (-1.5,1.5) grid (1.5,-1.5);
    \node[state, initial, accepting] (0) at (180:1.5) {};
    \node[state] (2) at (90:1.5) {};
    \node[state] (4) at (0:1.5) {};
    \node[state] (5) at (-45:1.5) {};
    \node[state] (6) at (-90:1.5) {};
    \node[state] (7) at (-135:1.5) {};

    \path
        (0) edge [loop above, left, rotate=20] node {$ab$} ()
        (0) edge [above, bend left = 20] node [above] {$a^2$} (2)
        (2) edge [above, bend left = 20] node [near start, below=+1mm] {$b^2$} (0)
        (2) edge [loop above] node [right=+1mm] {$ab+ba$} ()
        (2) edge [above, bend left = 20] node [above right=-1mm] {$a^2$} (4)
        (4) edge [above, bend left = 20] node [below left=-1.5mm] {$b^2$} (2)
        (4) edge [loop right] node {$ba$} ()
        (4) edge [above, bend left = 20] node [right] {$a$} (5)
        (5) edge [above, bend left = 20] node [above left=-1mm] {$b$} (4)
        (5) edge [above, bend left = 20] node [below] {$a$} (6)
        (6) edge [above, bend left = 20] node [above] {$b$} (5)
        (6) edge [above, bend left = 20] node [below] {$a$} (7)
        (7) edge [above, bend left = 20] node [above] {$b$} (6)
        (7) edge [above, bend left = 20] node [left] {$a$} (0)
        (0) edge [above, bend left = 20] node [above right=-1mm] {$b$} (7)
    ;
\end{tikzpicture}}\\
    \subfloat[][]{%
        \label{fig:automaton_R8_state_removal_steps-c}%
        \begin{tikzpicture}[my_automaton]
    \tikzstyle{every state}=[my_state]
    \tikzstyle{initial}=[my_initial_state]
    \tikzstyle{accepting}=[my_accepting_state]
    %\draw[help lines] (-1.5,1.5) grid (1.5,-1.5);
    \node[state, initial, accepting] (0) at (180:1.5) {};
    \node[state] (2) at (90:1.5) {};
    \node[state] (4) at (0:1.5) {};
    \node[state] (6) at (-90:1.5) {};
    \node[state] (7) at (-135:1.5) {};

    \path
        (0) edge [loop above, left, rotate=20] node {$ab$} ()
        (0) edge [above, bend left] node [above] {$a^2$} (2)
        (2) edge [above, bend left] node [near start, below=+1mm] {$b^2$} (0)
        (2) edge [loop above] node [right=+1mm] {$ab+ba$} ()
        (2) edge [above, bend left] node [above right=-1mm] {$a^2$} (4)
        (4) edge [above, bend left] node [below left=-1.5mm] {$b^2$} (2)
        (4) edge [loop right] node {$ab+ba$} ()
        (4) edge [above, bend left] node [right] {$a^2$} (6)
        (6) edge [above, bend left] node [above] {$b^2$} (4)
        (6) edge [loop below] node [right] {$ba$} ()
        (6) edge [above, bend left] node [below] {$a$} (7)
        (7) edge [above, bend left] node [above] {$b$} (6)
        (7) edge [above, bend left] node [left] {$a$} (0)
        (0) edge [above, bend left] node [above right=-1mm] {$b$} (7)
    ;
\end{tikzpicture}}%
    \hspace{8pt}%
    \subfloat[][]{%
        \label{fig:automaton_R8_state_removal_steps-d}%
        \begin{tikzpicture}[my_automaton]
    \tikzstyle{every state}=[my_state]
    \tikzstyle{initial}=[my_initial_state]
    \tikzstyle{accepting}=[my_accepting_state]
    %\draw[help lines] (-1.5,1.5) grid (1.5,-1.5);
    \node[state, initial, accepting] (0) at (180:1.5) {};
    \node[state] (2) at (90:1.5) {};
    \node[state] (4) at (0:1.5) {};
    \node[state] (6) at (-90:1.5) {};

    \path
        (0) edge [loop above, left, rotate=20] node {$ab+ba$} ()
        (0) edge [above, bend left = 20] node [above] {$a^2$} (2)
        (2) edge [above, bend left = 20] node [near start, below=+1mm] {$b^2$} (0)
        (2) edge [loop above] node [right=+1mm] {$ab+ba$} ()
        (2) edge [above, bend left = 20] node [above right=-1mm] {$a^2$} (4)
        (4) edge [above, bend left = 20] node [below left=-1.5mm] {$b^2$} (2)
        (4) edge [loop right] node {$ab+ba$} ()
        (4) edge [above, bend left = 20] node [right] {$a^2$} (6)
        (6) edge [above, bend left = 20] node [above] {$b^2$} (4)
        (6) edge [loop below] node [right] {$ab+ba$} ()
        (6) edge [above, bend left = 20] node [below left=-1.5mm] {$a^2$} (0)
        (0) edge [above, bend left = 20] node [above right=-1.5mm] {$b^2$} (6)
    ;
\end{tikzpicture}}\\
    \subfloat[][]{%
        \label{fig:automaton_R8_state_removal_steps-e}%
        \begin{tikzpicture}[my_automaton]
    \tikzstyle{every state}=[my_state]
    \tikzstyle{initial}=[my_initial_state]
    \tikzstyle{accepting}=[my_accepting_state]
    %\draw[help lines] (-1.5,1.5) grid (1.5,-1.5);
    \node[state, initial, initial above, accepting] (0) at (180:1.5) {};
    \node[state] (4) at (0:1.5) {};
    \node[state] (6) at (-90:1.5) {};

    \path
        (0) edge [loop left] node {$ab+ba+a^2{(ab+ba)}^*b^2$} ()
        (0) edge [above, bend left = 70] node [above] {$X_2$} (4)
        (4) edge [above, bend right = 50] node [below] {$Y_2$} (0)
        (4) edge [loop right] node {$ab+ba+b^2{(ab+ba)}^*a^2$} ()
        (4) edge [above, bend left = 20] node [right] {$a^2$} (6)
        (6) edge [above, bend left = 20] node [above] {$b^2$} (4)
        (6) edge [loop below] node [right] {$ab+ba$} ()
        (6) edge [above, bend left = 20] node [below left=-1mm] {$a^2$} (0)
        (0) edge [above, bend left = 20] node [above right=-1mm] {$b^2$} (6)
    ;
\end{tikzpicture}}\\
    \subfloat[][]{%
        \label{fig:automaton_R8_state_removal_steps-e}%
        \begin{tikzpicture}[my_automaton]
    \tikzstyle{every state}=[my_state]
    \tikzstyle{initial}=[my_initial_state]
    \tikzstyle{accepting}=[my_accepting_state]
    %\draw[help lines] (-1.5,1.5) grid (1.5,-1.5);
    \node[state, initial, initial above, accepting] (0) at (180:1.5) {};
    \node[state] (4) at (0:1.5) {};

    \path
        (0) edge [loop left] node {$Z_2$} ()
        (0) edge [bend left] node [above] {$X_2 + Y_2$} (4)
        (4) edge [bend left] node [below] {$X_2 + Y_2$} (0)
        (4) edge [loop right] node {$Z_2$} ()
    ;
\end{tikzpicture}}%
    \hspace{20pt}%
    \subfloat[][]{%
        \label{fig:automaton_R8_state_removal_steps-e}%
        \begin{tikzpicture}[->,>=latex,semithick]
    \tikzstyle{every state}=[circle, very thick, minimum size=3pt]
    \tikzstyle{initial}=[initial by arrow, initial text=]
    \tikzstyle{accepting}=[accepting by arrow]
    %\draw[help lines] (-1.5,1.5) grid (1.5,-1.5);
    \node[state, initial, initial above, accepting below] (0) at (180:1.5) {};

    \path
        (0) edge [loop right] node {$Z_2 + {(X_2 + Y_2)}Z_2^*{(X_2 + Y_2)}$} ()
    ;
\end{tikzpicture}}%
    \caption{First through seventh steps of the~state removal algorithm with order~$\omega$}%
    \label{fig:automaton_R8_state_removal_steps}%
\end{figure}

\begin{proof}[Proof of \autoref*{thm:main}]
\end{proof}