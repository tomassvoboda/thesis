\chapter{Eggan's question}

In this chapter we present an answer to the question whether there are rational languages with arbitrarily large star heights. The first examples of those languages are given by~\cite{Eggan63}. \X{Should be a citation?$\to$} Eggan's examples use an~alphabet of size $2^n - 1$ for the language with star height $n$. He therefore asked whether there are some examples of languages over binary alphabets. Answer to that question, which is affirmative, was provided shortly after by~\cite{DejeanSchutzenberger66}. Their answer is also reformulated by~\cite{Sakarovitch09}, which provided a basis for this work.

Throughout this chapter we use language $W_q = {\{f \mid |f|_a \equiv |f|_b \pmod{2^q} \}}$.

\begin{thm}\label{thm:main}
    The language $W_q$ has star height~$q$.
\end{thm}

The proof has two parts. First we find a~rational expression of star height~$q$, denoting $W_q$, and then we show that the star height of the language has to be at~least $q$. The first part is done as proof of Lemma~\ref*{lm:expression_existence}, which leaves the second part to be done in the Proof of~\autoref*{thm:main}.

\begin{lemma}\label{lm:expression_existence}
    Language $W_q$ is denoted by a rational expression of star height $q$.
\end{lemma}

\begin{proof}
    $W_q$ is recognised by an automaton~${\mathcal{R}(2^q)}$. The rational expression of star height $q$ will be obtained by applying the state removal algorithm on~${\mathcal{R}(2^q)}$. That means that the only transition left in the final automaton will be labeled by the desired rational expression. First we inductively define the following rational expressions:
    \begin{align*}
        \re{X}_0 &= a, & \re{Y}_0 &= b, & &\text{ and } & \re{Z}_0 &= 0, \\
        \re{X}_1 &= a^2, & \re{Y}_1 &= b^2, & &\text{ and } & \re{Z}_1 &= ab+ba, \\
        &\; \vdots & &\; \vdots & & & &\; \vdots \\
        \re{X}_{n+1} &= \re{X}_n \re{Z}_n^* \re{X}_n, & \re{Y}_{n+1} &= \re{Y}_n \re{Z}_n^* \re{Y}_n, & &\text{ and } & \re{Z}_{n+1} &= \re{Z}_n + \re{X}_n \re{Z}_n^* \re{Y}_n + \re{Y}_n \re{Z}_n^* \re{X}_n.
    \end{align*}

    Next, we look at constructing an~automaton, which we will call ${\mathcal{R}(2^q)}_0$, from ${\mathcal{R}(2^q)}$. The construction is shown in~\autoref*{fig:construction_of_R2q_0}. Every transition labelled~$a$ is replaced by one labelled $\re{X}_0$, every transition labelled~$b$ is replaced by one labelled $\re{Y}_0$, and a~loop labelled $\re{Z}_0$ is added to every state. Finally, we add two new distinct states in the same way as in the beginning of the state removal algorithm. It is clear, that ${\mathcal{R}(2^q)}_0$ accepts the same language as ${\mathcal{R}(2^q)}$.

    \begin{figure}[h]%
        \centering
        \subfloat[An automaton~${\mathcal{R}(2^q)}$]{%
            \begin{tikzpicture}[my_automaton]
    \tikzstyle{every state}=[my_named_state]
    \tikzstyle{initial}=[my_initial_state]
    \tikzstyle{accepting}=[my_accepting_state]
    %\draw[help lines] (-1.5,1.5) grid (1.5,-1.5);
    \node[state, initial, accepting] (0) at (180:1.5) {$0$};
    \node[state] (1) at (115:1.7) {$1$};
    \node[state, dashed] (2) at (60:2.2) {};
    \node[state, dashed] (2q-2) at (-60:2.2) {};
    \node[state] (2q-1) at (-115:1.7) {$2^q \scriptstyle - \displaystyle 1$};

    \path
        (0) edge [bend left] node [above left=-1.5mm] {$a$} (1)
        (1) edge [bend left] node [right] {$b$} (0)
        (1) edge [bend left, dashed] node [above] {$a$} (2)
        (2) edge [bend left, dashed] node [below right=-1mm] {$b$} (1)

        (2q-2) edge [bend left, dashed] node [below] {$a$} (2q-1)
        (2q-1) edge [bend left, dashed] node [above=-0.5mm] {$b$} (2q-2)
        (2q-1) edge [bend left] node [near start, left] {$a$} (0)
        (0) edge [bend left] node [right] {$b$} (2q-1)
    ;
\end{tikzpicture}}%
        \hspace{60pt}%
        \subfloat[An automaton~${\mathcal{R}(2^q)}_0$]{%
            \begin{tikzpicture}[my_automaton]
    \tikzstyle{every state}=[my_named_state]
    \tikzstyle{initial}=[my_initial_state]
    \tikzstyle{accepting}=[my_accepting_state]
    %\draw[help lines] (-1.5,1.5) grid (1.5,-1.5);
    \node[state] (0) at (180:1.5) {$0$};
    \node[state] (1) at (115:1.7) {$1$};
    \node[state, dashed] (2) at (60:2.2) {};
    \node[state, dashed] (2q-2) at (-60:2.2) {};
    \node[state] (2q-1) at (-115:1.7) {$2^q \scriptstyle - \displaystyle 1$};
    \node[state, initial, initial above] (i) at (155:3) {$i$};
    \node[state, accepting] (t) at (-155:3) {$t$};

    \path
        (0) edge [loop right] node {$\re{Z}_0$} ()
        (0) edge [bend left] node [above left=-1.5mm] {$\re{X}_0$} (1)
        (1) edge [bend left] node [right] {$\re{Y}_0$} (0)
        (1) edge [loop above, left] node {$\re{Z}_0$} ()
        (1) edge [bend left, dashed] node [above] {$\re{X}_0$} (2)
        (2) edge [bend left, dashed] node [below right=-1mm] {$\re{Y}_0$} (1)

        (2q-2) edge [bend left, dashed] node [below] {$\re{X}_0$} (2q-1)
        (2q-1) edge [bend left, dashed] node [above=-1mm] {$\re{Y}_0$} (2q-2)
        (2q-1) edge [in=-100, out=-80, distance=0.6cm, loop, left] node {$\re{Z}_0$} ()
        (2q-1) edge [bend left] node [near start, left=-1mm] {$\re{X}_0$} (0)
        (0) edge [bend left] node [right=-1mm] {$\re{Y}_0$} (2q-1)

        (i) edge [bend right] node [left] {$1$} (0)
        (0) edge [bend right] node [above left=-1mm] {$1$} (t)
    ;
\end{tikzpicture}}
        \caption{Construction of ${\mathcal{R}(2^q)}_0$ from ${\mathcal{R}(2^q)}$}\label{fig:construction_of_R2q_0}%
    \end{figure}

    This phase of the proof has $q-2$ steps. In each step $i$, for $1 \leq i \leq q-2$, we construct ${\mathcal{R}(2^q)}_i$ from ${\mathcal{R}(2^q)}_{i-1}$ without changing the recognised language. Each step $i$ of this phase comprises of $2^{q-i}$ steps of the state removal algorithm. The states removed are all the odd states of the ${\mathcal{R}(2^q)}_{i-1}$ automaton. That means that if ${\mathcal{R}(2^q)}$ has $2^q$ states, then ${\mathcal{R}(2^q)}_0$ has $2^q + 2$ states, ${\mathcal{R}(2^q)}_1$ has $2^{q-1} + 2$ states, and so on, which leaves the automaton produced by the last step of this phase, ${\mathcal{R}(2^q)}_{q-2}$, with $2^{q-(q-2)} + 2 = 6$ states. After the removal of the $2^{q-i}$ odd states, the remaining $2^{q-i}$ even states, that is excluding the two special states~$i$, and~$t$, are renamed as members of $\Z_{2^{q-i}}$. \autoref*{fig:construction_of_R2q_i} illustrates the $i$-th step of this phase.
    \X{Can the figure below overflow the text margins?}

    \begin{figure}[h]%
        \centerline{
            \hspace{-25pt}%
            \subfloat[An automaton~${\mathcal{R}(2^q)}_{i-1}$]{%
                \begin{tikzpicture}[my_automaton]
    \tikzstyle{every state}=[my_named_state]
    \tikzstyle{initial}=[my_initial_state]
    \tikzstyle{accepting}=[my_accepting_state]
    %\draw[help lines] (-1.5,1.5) grid (1.5,-1.5);
    \node[state] (0) at (180:1.5) {$0$};
    \node[state] (1) at (115:1.7) {$1$};
    \node[state, dashed] (2) at (60:2.2) {};
    \node[state, dashed] (2q-2) at (-50:2.4) {};
    \node[state, inner sep=0.01cm] (2q-1) at (-105:1.9) [draw, align=center] {$2^{q-(i-1)}$ \\ $\scriptstyle - \displaystyle 1$};
    \node[state, initial] (i) at (155:3) {$i$};
    \node[state, accepting] (t) at (-155:3) {$t$};

    \path
        (0) edge [loop right] node {$\re{Z}_{i-1}$} ()
        (0) edge [bend left] node [near end, above left=-1.5mm] {$\re{X}_{i-1}$} (1)
        (1) edge [bend left] node [right] {$\re{Y}_{i-1}$} (0)
        (1) edge [loop above, left] node {$\re{Z}_{i-1}$} ()
        (1) edge [bend left, dashed] node [above] {$\re{X}_{i-1}$} (2)
        (2) edge [bend left, dashed] node [below right=-1mm] {$\re{Y}_{i-1}$} (1)

        (2q-2) edge [bend left, dashed] node [below] {$\re{X}_{i-1}$} (2q-1)
        (2q-1) edge [bend left, dashed] node [above=-1mm] {$\re{Y}_{i-1}$} (2q-2)
        (2q-1) edge [in=-100, out=-80, distance=0.6cm, loop, left] node {$\re{Z}_{i-1}$} ()
        (2q-1) edge [bend left] node [near start, left=-1mm] {$\re{X}_{i-1}$} (0)
        (0) edge [bend left] node [near end, right=-1mm] {$\re{Y}_{i-1}$} (2q-1)

        (i) edge [bend right] node [left] {$1$} (0)
        (0) edge [bend right] node [above left=-1mm] {$1$} (t)
    ;
\end{tikzpicture}}%
            \quad
            \subfloat[${\mathcal{R}(2^q)}_{i-1}$ after $2^{q-i}$ steps of the state removal algorithm]{\label{fig:odd_states_removed}%
                \begin{tikzpicture}[my_automaton]
    \tikzstyle{every state}=[my_named_state]
    \tikzstyle{initial}=[my_initial_state]
    \tikzstyle{accepting}=[my_accepting_state]
    %\draw[help lines] (-1.5,1.5) grid (1.5,-1.5);
    \node[state] (0) at (180:1.5) {$0$};
    \node[state] (1) at (115:1.7) {$2$};
    \node[state, dashed] (2) at (60:2.2) {};
    \node[state, dashed] (2q-2) at (-50:2.4) {};
    \node[state, inner sep=0.01cm] (2q-1) at (-105:1.9) [draw, align=center] {$2^{q-(i-1)}$ \\ $\scriptstyle - \displaystyle 2$};
    \node[state, initial, initial above] (i) at (155:3) {$i$};
    \node[state, accepting] (t) at (-155:3) {$t$};

    \path
        (0) edge [loop right] node {$\re{Z}_i$} ()
        (0) edge [bend left] node [above left=-1.5mm] {$\re{X}_i$} (1)
        (1) edge [bend left] node [right] {$\re{Y}_i$} (0)
        (1) edge [loop above, left] node {$\re{Z}_i$} ()
        (1) edge [bend left, dashed] node [above] {$\re{X}_i$} (2)
        (2) edge [bend left, dashed] node [below right=-1mm] {$\re{Y}_i$} (1)

        (2q-2) edge [bend left, dashed] node [below] {$\re{X}_i$} (2q-1)
        (2q-1) edge [bend left, dashed] node [above=-1mm] {$\re{Y}_i$} (2q-2)
        (2q-1) edge [in=-100, out=-80, distance=0.6cm, loop, left] node {$\re{Z}_i$} ()
        (2q-1) edge [bend left] node [near start, left=-1mm] {$\re{X}_i$} (0)
        (0) edge [bend left] node [right=-1mm] {$\re{Y}_i$} (2q-1)

        (i) edge [bend right] node [left] {$1$} (0)
        (0) edge [bend right] node [above left=-1mm] {$1$} (t)
    ;
\end{tikzpicture}}
            \quad
            \subfloat[Automaton from figure~\ref*{fig:odd_states_removed} with renamed states, which is an automaton~${\mathcal{R}(2^q)}_i$]{%
                \begin{tikzpicture}[my_automaton]
    \tikzstyle{every state}=[my_named_state]
    \tikzstyle{initial}=[my_initial_state]
    \tikzstyle{accepting}=[my_accepting_state]
    %\draw[help lines] (-1.5,1.5) grid (1.5,-1.5);
    \node[state] (0) at (180:1.5) {$0$};
    \node[state] (1) at (115:1.7) {$1$};
    \node[state, dashed] (2) at (60:2.2) {};
    \node[state, dashed] (2q-2) at (-50:2.4) {};
    \node[state, inner sep=0.04cm] (2q-1) at (-105:1.9) [draw, align=center] {$2^{q-i}$ \\ $\scriptstyle - \displaystyle 1$};
    \node[state, initial] (i) at (155:3) {$i$};
    \node[state, accepting] (t) at (-155:3) {$t$};

    \path
        (0) edge [loop right] node {$\re{Z}_i$} ()
        (0) edge [bend left] node [above left=-1.5mm] {$\re{X}_i$} (1)
        (1) edge [bend left] node [right] {$\re{Y}_i$} (0)
        (1) edge [loop above, left] node {$\re{Z}_i$} ()
        (1) edge [bend left, dashed] node [above] {$\re{X}_i$} (2)
        (2) edge [bend left, dashed] node [below right=-1mm] {$\re{Y}_i$} (1)

        (2q-2) edge [bend left, dashed] node [below] {$\re{X}_i$} (2q-1)
        (2q-1) edge [bend left, dashed] node [above=-1mm] {$\re{Y}_i$} (2q-2)
        (2q-1) edge [in=-100, out=-80, distance=0.6cm, loop, left] node {$\re{Z}_i$} ()
        (2q-1) edge [bend left] node [near start, left=-1mm] {$\re{X}_i$} (0)
        (0) edge [bend left] node [right=-1mm] {$\re{Y}_i$} (2q-1)

        (i) edge [bend right] node [left] {$1$} (0)
        (0) edge [bend right] node [above left=-1mm] {$1$} (t)
    ;
\end{tikzpicture}}%
        }
        \caption{Construction of ${\mathcal{R}(2^q)}_i$ from ${\mathcal{R}(2^q)}_{i-1}$}\label{fig:construction_of_R2q_i}%
    \end{figure}

    The result of the $q-2$ steps described above is an automaton~${\mathcal{R}(2^q)}_{q-2}$, shown in Figure~\ref*{fig:automaton_R2q_q-2}. ${\mathcal{R}(2^q)}_{q-1}$ is clearly a result of the states $1$, and $3$ being removed from the automaton~${\mathcal{R}(2^q)}_{q-2}$ by the state removal algorithm. The equivalence
    \[
        {(\re{Z}_{q-1} + {(\re{X}_{q-1} + \re{Y}_{q-1})}\re{Z}_{q-1}^*{(\re{X}_{q-1} + \re{Y}_{q-1})})}^* = {(\re{X}_q + \re{Y}_q + \re{Z}_q)}^*
    \]
    shows, that after applying the state removal algorithm on the state $1$ of ${\mathcal{R}(2^q)}_{q-1}$, we get ${\mathcal{R}(2^q)}_{q}$. These two last steps are shown in the~\autoref*{fig:construction_of_R2q_q}. \X{Fix the vertical space of the subfigures!}

    \begin{figure}[h]%
        \centerline{
            \hspace{-25pt}%
            \subfloat[An automaton~${\mathcal{R}(2^q)}_{q-2}$]{\label{fig:automaton_R2q_q-2}%
                \begin{tikzpicture}[my_automaton]
    \tikzstyle{every state}=[my_named_state]
    \tikzstyle{initial}=[my_initial_state]
    \tikzstyle{accepting}=[my_accepting_state]
    %\draw[help lines] (-1.5,1.5) grid (1.5,-1.5);
    \node[state] (0) at (180:1.8) {$0$};
    \node[state] (1) at (90:1.8) {$1$};
    \node[state] (2) at (0:1.8) {$2$};
    \node[state] (3) at (-90:1.8) {$3$};

    \node[state, initial, initial above] (i) at (155:3) {$i$};
    \node[state, accepting] (t) at (-155:3) {$t$};

    \path
        (0) edge [loop left] node {$\re{Z}_{q-2}$} ()
        (0) edge [bend left] node [above left=-1.5mm] {$\re{X}_{q-2}$} (1)
        (1) edge [bend left=10] node [near end, below right=-1.5mm] {$\re{Y}_{q-2}$} (0)
        (1) edge [loop above] node [right=+1mm] {$\re{Z}_{q-2}$} ()
        (1) edge [bend left] node [above right=-2mm] {$\re{X}_{q-2}$} (2)
        (2) edge [bend left=10] node [below left=-3mm] {$\re{Y}_{q-2}$} (1)
        (2) edge [loop left] node [left=-1mm] {$\re{Z}_{q-2}$} ()
        (2) edge [bend left] node [near end, below right=-1mm] {$\re{X}_{q-2}$} (3)
        (3) edge [bend left=10] node [above left=-2mm] {$\re{Y}_{q-2}$} (2)
        (3) edge [loop below] node [right=+1mm] {$\re{Z}_{q-2}$} ()
        (3) edge [bend left] node [near start, below left=-3mm] {$\re{X}_{q-2}$} (0)
        (0) edge [bend left=10] node [near start, above right=-3mm] {$\re{Y}_{q-2}$} (3)

        (i) edge [bend left] node [left] {$\re{1}$} (0)
        (0) edge [bend left] node [above left=-1mm] {$\re{1}$} (t)
    ;
\end{tikzpicture}}%
            \quad
            \subfloat[An automaton~${\mathcal{R}(2^q)}_{q-1}$]{%
                \begin{tikzpicture}[my_automaton]
    \tikzstyle{every state}=[my_named_state]
    \tikzstyle{initial}=[my_initial_state]
    \tikzstyle{accepting}=[my_accepting_state]
    %\draw[help lines] (-1.5,1.5) grid (1.5,-1.5);
    \node[state] (0) at (180:1.4) {$0$};
    \node[state] (1) at (0:1.4) {$1$};

    \node[state, initial, initial above] (i) at (145:2.7) {$i$};
    \node[state, accepting] (t) at (-145:2.7) {$t$};

    \path
        (0) edge [loop left] node [left=-1mm] {$\re{Z}_{q-1}$} ()
        (0) edge [bend left] node [above=-1mm] {$\re{X}_{q-1} + \re{Y}_{q-1}$} (1)
        (1) edge [bend left] node [below=-0.5mm] {$\re{X}_{q-1} + \re{Y}_{q-1}$} (0)
        (1) edge [loop above] node {$\re{Z}_{q-1}$} ()

        (i) edge [bend left] node [left] {$\re{1}$} (0)
        (0) edge [bend left] node [above left=-1mm] {$\re{1}$} (t)
    ;
\end{tikzpicture}}%
            \quad
            \subfloat[An automaton~${\mathcal{R}(2^q)}_{q}$]{%
                \begin{tikzpicture}[my_automaton]
    \tikzstyle{every state}=[my_named_state]
    \tikzstyle{initial}=[my_initial_state]
    \tikzstyle{accepting}=[my_accepting_state]
    %\draw[help lines] (-1.5,1.5) grid (1.5,-1.5);
    \node[state] (0) at (180:1.4) {$0$};

    \node[state, initial] (i) at (145:2.7) {$i$};
    \node[state, accepting] (t) at (-145:2.7) {$t$};

    \path
        (0) edge [loop right] node {$\re{X}_{q} + \re{Y}_{q} + \re{Z}_{q}$} ()

        (i) edge [bend left] node [left] {$1$} (0)
        (0) edge [bend left] node [above left=-1mm] {$1$} (t)
    ;
\end{tikzpicture}}%
        }
        \caption{Steps $q-2, \: q-1$, and $q$ of the construction of ${\mathcal{R}(2^q)}_q$}\label{fig:construction_of_R2q_q}%
    \end{figure}

    Each of the rational expressions $\re{X}_q, \re{Y}_q$, and $\re{Z}_q$ has star height of $q-1$, therefore ${(\re{X}_q + \re{Y}_q + \re{Z}_q)}^*$ has star height of $q$. Since none of the automaton constructions changed the recognised language, it follows that $W_q$ is denoted by ${(\re{X}_q + \re{Y}_q + \re{Z}_q)}^*$.
\end{proof}

\begin{example}
    According to the procedure described in the Proof of~Lemma~\ref*{lm:expression_existence}, we apply the state removal algorithm on the automaton~${\mathcal{R}(8)}$, recognizing language~$W_3$. The steps of the creation of rational expression ${(\re{X}_3 + \re{Y}_3 + \re{Z}_3)}^*$, denoting $W_3$, are shown in~\autoref*{fig:automaton_R8_state_removal_steps}. \X{Fix the automatons in the figure!}

    \begin{figure}%
        \centering
        \hspace{-20pt}%
        \subfloat[][]{\label{fig:automaton_R8_state_removal_steps-a}%
            \begin{tikzpicture}[->,>=latex,semithick]
    \tikzstyle{every state}=[circle, very thick, minimum size=3pt]
    \tikzstyle{initial}=[initial by arrow, initial text=]
    \tikzstyle{accepting}=[accepting by arrow]
    %\draw[help lines] (-1.5,1.5) grid (1.5,-1.5);
    \node[state, initial, initial, accepting] (0) at (180:1.5) {};
    \node[state] (2) at (90:1.5) {};
    \node[state] (3) at (45:1.5) {};
    \node[state] (4) at (0:1.5) {};
    \node[state] (5) at (-45:1.5) {};
    \node[state] (6) at (-90:1.5) {};
    \node[state] (7) at (-135:1.5) {};

    \path
        (0) edge [loop above, left, rotate=20] node {$ab$} ()
        (0) edge [above, bend left = 20] node [above] {$a^2$} (2)
        (2) edge [above, bend left = 20] node [near start, below=+1mm] {$b^2$} (0)
        (2) edge [loop above] node {$ba$} ()
        (2) edge [above, bend left = 20] node [above right] {$a$} (3)
        (3) edge [above, bend left = 20] node [near start, below left] {$b$} (2)
        (3) edge [above, bend left = 20] node [right] {$a$} (4)
        (4) edge [above, bend left = 20] node [left] {$b$} (3)
        (4) edge [above, bend left = 20] node [right] {$a$} (5)
        (5) edge [above, bend left = 20] node [above left] {$b$} (4)
        (5) edge [above, bend left = 20] node [below] {$a$} (6)
        (6) edge [above, bend left = 20] node [above] {$b$} (5)
        (6) edge [above, bend left = 20] node [below] {$a$} (7)
        (7) edge [above, bend left = 20] node [above] {$b$} (6)
        (7) edge [above, bend left = 20] node [left] {$a$} (0)
        (0) edge [above, bend left = 20] node [right] {$b$} (7)
    ;
\end{tikzpicture}}%
        \hspace{70pt}%
        \subfloat[][]{\label{fig:automaton_R8_state_removal_steps-b}%
            \begin{tikzpicture}[my_automaton]
    \tikzstyle{every state}=[my_state]
    \tikzstyle{initial}=[my_initial_state]
    \tikzstyle{accepting}=[my_accepting_state]
    %\draw[help lines] (-1.5,1.5) grid (1.5,-1.5);
    \node[state, initial, accepting] (0) at (180:1.5) {};
    \node[state] (2) at (90:1.5) {};
    \node[state] (4) at (0:1.5) {};
    \node[state] (5) at (-45:1.5) {};
    \node[state] (6) at (-90:1.5) {};
    \node[state] (7) at (-135:1.5) {};

    \path
        (0) edge [loop above, left, rotate=20] node {$ab$} ()
        (0) edge [above, bend left] node [above] {$a^2$} (2)
        (2) edge [above, bend left] node [near start, below=+1mm] {$b^2$} (0)
        (2) edge [loop above] node [right=+1mm] {$ab+ba$} ()
        (2) edge [above, bend left] node [above right=-1mm] {$a^2$} (4)
        (4) edge [above, bend left] node [below left=-1.5mm] {$b^2$} (2)
        (4) edge [loop right] node {$ba$} ()
        (4) edge [above, bend left] node [right] {$a$} (5)
        (5) edge [above, bend left] node [above left=-1mm] {$b$} (4)
        (5) edge [above, bend left] node [below] {$a$} (6)
        (6) edge [above, bend left] node [above] {$b$} (5)
        (6) edge [above, bend left] node [below] {$a$} (7)
        (7) edge [above, bend left] node [above] {$b$} (6)
        (7) edge [above, bend left] node [left] {$a$} (0)
        (0) edge [above, bend left] node [above right=-1mm] {$b$} (7)
    ;
\end{tikzpicture}}\\
        \subfloat[][]{\label{fig:automaton_R8_state_removal_steps-c}%
            \begin{tikzpicture}[my_automaton]
    \tikzstyle{every state}=[my_state]
    \tikzstyle{initial}=[my_initial_state]
    \tikzstyle{accepting}=[my_accepting_state]
    %\draw[help lines] (-1.5,1.5) grid (1.5,-1.5);
    \node[state, initial, accepting] (0) at (180:1.5) {};
    \node[state] (2) at (90:1.5) {};
    \node[state] (4) at (0:1.5) {};
    \node[state] (6) at (-90:1.5) {};
    \node[state] (7) at (-135:1.5) {};

    \path
        (0) edge [loop above, left, rotate=20] node {$ab$} ()
        (0) edge [bend left] node [above] {$a^2$} (2)
        (2) edge [bend left] node [near start, below=+1mm] {$b^2$} (0)
        (2) edge [loop above] node [right=+1mm] {$ab+ba$} ()
        (2) edge [bend left] node [above right=-1mm] {$a^2$} (4)
        (4) edge [bend left] node [below left=-1.5mm] {$b^2$} (2)
        (4) edge [loop right] node {$ab+ba$} ()
        (4) edge [bend left] node [right] {$a^2$} (6)
        (6) edge [bend left] node [above] {$b^2$} (4)
        (6) edge [loop below] node [right] {$ba$} ()
        (6) edge [bend left] node [below] {$a$} (7)
        (7) edge [bend left] node [above] {$b$} (6)
        (7) edge [bend left] node [left] {$a$} (0)
        (0) edge [bend left] node [above right=-1mm] {$b$} (7)
    ;
\end{tikzpicture}}%
        \hspace{8pt}%
        \subfloat[][]{\label{fig:automaton_R8_state_removal_steps-d}%
            \begin{tikzpicture}[my_automaton]
    \tikzstyle{every state}=[my_state]
    \tikzstyle{initial}=[my_initial_state]
    \tikzstyle{accepting}=[my_accepting_state]
    %\draw[help lines] (-1.5,1.5) grid (1.5,-1.5);
    \node[state, initial, accepting] (0) at (180:1.5) {};
    \node[state] (2) at (90:1.5) {};
    \node[state] (4) at (0:1.5) {};
    \node[state] (6) at (-90:1.5) {};

    \path
        (0) edge [loop above, left, rotate=20] node {$ab+ba$} ()
        (0) edge [bend left] node [above] {$a^2$} (2)
        (2) edge [bend left] node [near start, below=+1mm] {$b^2$} (0)
        (2) edge [loop above] node [right=+1mm] {$ab+ba$} ()
        (2) edge [bend left] node [above right=-1mm] {$a^2$} (4)
        (4) edge [bend left] node [below left=-1.5mm] {$b^2$} (2)
        (4) edge [loop right] node {$ab+ba$} ()
        (4) edge [bend left] node [right] {$a^2$} (6)
        (6) edge [bend left] node [above] {$b^2$} (4)
        (6) edge [loop below] node [right] {$ab+ba$} ()
        (6) edge [bend left] node [below left=-1.5mm] {$a^2$} (0)
        (0) edge [bend left] node [above right=-1.5mm] {$b^2$} (6)
    ;
\end{tikzpicture}}\\
        \subfloat[][]{\label{fig:automaton_R8_state_removal_steps-e}%
            \begin{tikzpicture}[my_automaton]
    \tikzstyle{every state}=[my_state]
    \tikzstyle{initial}=[my_initial_state]
    \tikzstyle{accepting}=[my_accepting_state]
    %\draw[help lines] (-1.5,1.5) grid (1.5,-1.5);
    \node[state, initial, initial above, accepting] (0) at (180:1.5) {};
    \node[state] (4) at (0:1.5) {};
    \node[state] (6) at (-90:1.5) {};

    \path
        (0) edge [loop left] node {$ab+ba+a^2{(ab+ba)}^*b^2$} ()
        (0) edge [above, bend left = 70] node [above] {$X_2$} (4)
        (4) edge [above, bend right = 50] node [below] {$Y_2$} (0)
        (4) edge [loop right] node {$ab+ba+b^2{(ab+ba)}^*a^2$} ()
        (4) edge [above, bend left] node [right] {$a^2$} (6)
        (6) edge [above, bend left] node [above] {$b^2$} (4)
        (6) edge [loop below] node [right] {$ab+ba$} ()
        (6) edge [above, bend left] node [below left=-1mm] {$a^2$} (0)
        (0) edge [above, bend left] node [above right=-1mm] {$b^2$} (6)
    ;
\end{tikzpicture}}\\
        \subfloat[][]{\label{fig:automaton_R8_state_removal_steps-f}%
            \begin{tikzpicture}[my_automaton]
    \tikzstyle{every state}=[my_state]
    \tikzstyle{initial}=[my_initial_state]
    \tikzstyle{accepting}=[my_accepting_state]
    %\draw[help lines] (-1.5,1.5) grid (1.5,-1.5);
    \node[state, initial, initial above, accepting] (0) at (180:1.5) {};
    \node[state] (4) at (0:1.5) {};

    \path
        (0) edge [loop left] node {$Z_2$} ()
        (0) edge [above, bend left] node [above] {$X_2 + Y_2$} (4)
        (4) edge [above, bend left] node [below] {$X_2 + Y_2$} (0)
        (4) edge [loop right] node {$Z_2$} ()
    ;
\end{tikzpicture}}%
        \hspace{40pt}%
        \subfloat[][]{\label{fig:automaton_R8_state_removal_steps-g}%
            \begin{tikzpicture}[my_automaton]
    \tikzstyle{every state}=[my_state]
    \tikzstyle{initial}=[my_initial_state]
    \tikzstyle{accepting}=[my_accepting_state]
    %\draw[help lines] (-1.5,1.5) grid (1.5,-1.5);
    \node[state, initial, initial above, accepting] (0) at (180:1.5) {};

    \path
        (0) edge [loop right] node {$Z_2 + {(X_2 + Y_2)}Z_2^*{(X_2 + Y_2)}$} ()
    ;
\end{tikzpicture}}%
        \caption{First through seventh steps of the~state removal algorithm with order~$\omega$}\label{fig:automaton_R8_state_removal_steps}%
    \end{figure}
\end{example}

We have found a~rational expression of star height $q$ denoting the language $W_q$, which means that the star height of the language will not be higher then $q$. We proceed to show that it also has to be at least $q$. First let us present a few definitions and observe some of their properties that will be useful to us later.

\begin{defn}
    For~each integer~$n$ we define the~sequence $w_{0,n}, w_{1,n}, \dotsc , w_{q-1,n}$ of $q$~words in~$W_q$ recursively:
    \begin{alignat*}{2}
        w_{0,n} &= ab,\\
        w_{1,n} &= a^2{(ab)}^{n}b^2{(ab)}^{n},\\
                &\; \vdots \\
        w_{k,n} &= a^{2^k}{(w_{k-1,n})}^{n}b^{2^k}{(w_{k-1,n})}^{n},\\
                &\; \vdots \\
        w_{q-1,n} &= a^{2^{q-1}}{(w_{q-2,n})}^{n}b^{2^{q-1}}{(w_{q-2,n})}^{n}.\\
    \end{alignat*}
    We call these words \emph{witness words}.
\end{defn}

\begin{lemma}\label{lm:witness_words_inequalities}
    Any left factor $u$ and right factor $v$ of ${(w_{k,n})}^n$ satisfy the equations:
    \begin{equation}
        0 \leq |u|_a - |u|_b \leq 2^{k+1}-1 \; , \qquad 0 \leq |v|_b - |v|_a \leq 2^{k+1}-1.
    \end{equation}
\end{lemma}

\begin{proof}
    $|w_{k,n}|_a = |w_{k,n}|_b$, therefore any whole word $w_{k,n}$ that is a factor of either $u$ or $v$ does not affect the inequalities we are proving. We only need to examine the differences in letters of the left (resp.~right) factor of the word $w_{k,n}$. We proceed by indunction on~$k$. Trivially the inequalities hold for $w_{0,n}$.
    \[
        w_{k,n} = a^{2^k}{(w_{k-1,n})}^{n}b^{2^k}{(w_{k-1,n})}^{n}
    \] begins with $2^k$ $a$'s and from the induction hypothesis, we know that $w_{k-1,n}$ can have at most $2^k-1$ more $a$'s than $b$'s. That proves the case of the left factor.
    Similarly for the case of the right factor $v$. From the induction hypothesis we know that the right factor of $w_{k-1,n}$ can have at most $2^k-1$ more $b$'s than $a$'s. Therefore if $b^{2^k}{(w_{k-1,n})}^{n}$ is a right factor of $v$, the right factor of ${(w_{k-1,n})}^{n}$ is the left factor of $v$. We can see that $|v|_b - |v|_a \leq 2^k + 2^k -1$.
\end{proof}

Next we use the witness words to define a property of languages.

\begin{defn}
    We say that language $L$ \emph{satisfies property $(P_k)$} if there exists an infinite number of values of $n$ such that ${(w_{k,n})}^n$ is a~factor of at~least one word in~L.
\end{defn}

Note that if $L$ satisfies $(P_k)$, it also satisfies $(P_l)$ for $l \leq k$ since ${(w_{k-1,n})}^n$ is a factor of ${(w_{k,n})}^n$.

\begin{proof}[Proof of \autoref*{thm:main}]
    Lemma~\ref*{lm:expression_existence} gives us an~expression with star height~$q$ denoting the~language~$W_q$. Now we show that the~height has to be at~least~$q$.\\
    By $\mathfrak{W}_k$ we denote a family of languages $L$ that satisfy the following conditions:
    \begin{enumerate}
        \item[(i)] $L \subseteq W_q$,
        \item[(ii)] $L$ satisfies $(P_k)$,
        \item[(iii)] $L$ has a minimum star height of any language satisfying the first two conditions.
    \end{enumerate}
    Let $h_k$ be the common value of the star height of the languages in $\mathfrak{W}_k$, then we have
    \[
        0 < h_0 \leq h_1 \leq \dotsb \leq h_{q-1} \leq q.
    \]
    Language in $\mathfrak{W}_0$ is necessarily infinite, therefore $0 < h_0$. Since $(P_k)$ implies $(P_{k-1})$, $h_{k-1} \leq h_k$. $W_q$ obviously satisfies (i), but it also satisfies $(P_{q-1})$ because for each $n$, $|{(w_{q-1,n})}^n|_a = |{(w_{q-1,n})}^n|_b$, and because in Lemma~\ref*{lm:expression_existence} we have found an expression of star height $q$, to satisfy the condition (iii), $h_{q-1} \leq q$.

    To prove the theorem it is enough to show that
    \[
        \forall \, k \in \{1,\dotsc,q-1\}: h_{k-1} < h_k.
    \]
    Let $L \in \mathfrak{W}_k$ with star height $h_k$. Due to Lemma~\ref*{lm:distributivity} $L$ can be written as a finite union of languages $E_j$, each denoted by a expression of the form:
    \[
        \mathsf{E}_j = u_0{(\mathsf{H}_1)}^*u_1 \dotsm u_{m-1}{(\mathsf{H}_m)}^*u_m.
    \]
    Each expression $\mathsf{H}_i$ denotes a~language~$H_i$ with star height less then or equal to $h_k-1$. At least one of the languages $E_j$ has to satisfy $(P_k)$ since $L$ satisfies it and there in only a finite number of languages $E_j$ in the union equal to $L$. It is therefore safe to assume that $L$ is denoted by an expression of the same form as $E_j$.

    Each of the $H_i$ must be contained in $W_q$. For contradiction, let us have word $g \in H_i$, that $|g|_a \not\equiv |g|_b \pmod{2^q}$. Then it must be true that both
    \[
        u_0 u_1 \dotsm u_m \in L \qquad \text{and} \qquad u_0 u_1 \dotsm u_{i-1} g u_{i+1} \dotsm u_m \in L,
    \]
    which contradicts the fact that $\forall f \in L: |f|_a \equiv |f|_b \pmod{2^q}$.

    Since $L$ is of a minimal star height to satisfy $(P_k)$, none of the $H_i$ satisfies $(P_k)$. Now we need to show that some $H_i$ satisfies $(P_{k-1})$. In fact we will have $h_{k-1} \leq h_k - 1$.

    $L$ satisfies $(P_k)$, so for arbitrarily large $n$, words ${(w_{k,n})}^n$ are factors of $L$. \X{Meaning that $\forall n \; \exists f \in L: {(w_{k,n})}^n$ is a factor of $f$? Hopefully.} Lemma~\ref*{lm:block_star_lemma} shows, that we can find $N$ large enough, that ${(w_{k,N})}^N$ will be a factor of $f \in L$, which still gives us infinitely many $n' \geq N$, that ${(w_{k,n'})}^{n'}$ is a factor some word, and for each of those $n'$ we have infinitely many $l$'s, that ${(w_{k,n})}^l$ is a factor of $L$\X{?}, again due to Lemma~\ref*{lm:block_star_lemma}. Since $m$ is finite and fixed for $L$, there has to be infinitely many $n$'s, that ${(w_{k,n})}^n$ is a factor a word $r_n \in H_i^*$, for some $i$, $1 \leq i \leq m$, therefore $H_i^*$ satisfies $(P_k)$. We will show that for these $n$'s, ${(w_{k-1,n})}^n$ is a factor of a word in $H_i$, meaning $H_i$ satisfies $(P_{k-1})$.

    We write $r_n$ as a factorization $r_n = g_0 g_1 \dotsm g_l$, where $g_j \in H_i$, for $0 \leq j \leq l$. If $w_{k,n}$, from ${(w_{k,n})}^n$, is a factor of some $g_j$, the condition is satisfied, since ${(w_{k-1,n})}^n$ is a factor of $w_{k,n}$. Otherwise we will examine how a factor ${(w_{k,n})}^2$ is covered by the factorization of $r_n$. Written explicitly, we have
    \[
        {(w_{k,n})}^2 = a^{2^k}{(w_{k-1,n})}^{n}b^{2^k}{(w_{k-1,n})}^{n}a^{2^k}{(w_{k-1,n})}^{n}b^{2^k}{(w_{k-1,n})}^{n}.
    \]
    Let us consider $g_j$, that the leftmost factor $b^{2^k}$ is covered by it, at least partially. There are two possibilities:
    \begin{enumerate}
        \item[(i)] $b^{2^k}$ is a factor of $g_j$,
        \item[(ii)] a left factor of $b^{2^k}$ is a right factor of $g_j$.
    \end{enumerate}

    In the case (i), we have $g_j = v b^{2^k} u$. If $v$ covers the factor ${(w_{k-1,n})}^{n}$ to the left of $b^{2^k}$, or $u$ covers the factor ${(w_{k-1,n})}^{n}$ to the right of $b^{2^k}$, the condition is satisfied. If not, $u$ is a left factor of ${(w_{k-1,n})}^{n}$ and $v$ is a right factor. Set
    \[
        x = |g_j|_b - |g_j|_a, \quad y = |u|_a - |u|_b, \quad \text{ and } \quad z = |v|_b - |v|_a.
    \]
    Hence
    \[
        x = 2^k - (y - z).
    \]
    We see that $1 - 2^k \leq y - z \leq 2^k - 1$, from Lemma~\ref*{lm:witness_words_inequalities}. That gives us
    \[
        0 < x < 2^{k+1} \leq 2^q.
    \]
    which contradicts the fact that $g_j \in H_i \subseteq W_q$.

    In the case (ii) we have $g_j = v b^{r}, \: 0 < r < 2^k$. If ${(w_{k-1,n})}^{n}$ is a factor of $v$, the condition is satisfied. Otherwise $v$ is a right factor of ${(w_{k-1,n})}^{n}$ and, similarly as above, we set
    \[
        x = |g_j|_b - |g_j|_a, \quad \text{ and } \quad z = |v|_b - |v|_a.
    \]
    Hence
    \[
        x = r + z.
    \]
    Therefore, due to Lemma~\ref*{lm:witness_words_inequalities},
    \[
        r \leq x \leq r + 2^k - 1 < 2^{k+1} < 2^q.
    \]
    which produces the same contradiction as the case (i).
\end{proof}