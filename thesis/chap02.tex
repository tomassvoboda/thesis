\chapter{Eggan's question}

\cite{Eggan63} asks whether there are languages with arbitrarily large star height. In this section we present proof first due to~\cite{DejeanSchutzenberger66} and recently formulated in~\cite{Sakarovitch09}.

Throughout this chapter we use language $W_q = {\{f \mid |f|_a \equiv |f|_b \pmod{2^q} \}}$.

\begin{thm}\label{thm:main}
    The language $W_q$ has star height~$q$.
\end{thm}

For the~proof we need to find an~expression of star height~$q$ denoting~$W_q$. That is done in~Lemma~\ref*{lm:expression_existence}. Then it remains to show that $W_q$ has the star height of at~least~$q$.

\begin{lemma}\label{lm:expression_existence}
    For each $q \in \N$ there is an expression of star height~$q$, that denotes~$W_q$.
\end{lemma}

\begin{proof}
    $W_q$ is recognised by an automaton~${\mathcal{R}(2^q)}$. By following a specific order~$\omega$ in the elimination method on $W_q$ we get an expression of a~star height $q$. First we set expressions
    \begin{alignat*}{5}
        X_1 = a^2 , \quad Y_1 = b^2, \quad &\text{ and } \quad Z_1 = ab + ba.
    \intertext{Then for every integer $n$ we have}
        X_{n+1} = X_n Z_n^* X_n , \quad Y_{n+1} = Y_n Z_n^* Y_n, \quad &\text{ and } \quad Z_{n+1} = Z_n + X_n Z_n^* Y_n + Y_n Z_n^* X_n.
    \end{alignat*}
    It follows that
    \[
        W_q = {(X_q + Y_q + Z_q)}^*.
    \]
    Since $X_n , Y_n , \text{ and } Z_n$ have star height~$n~-~1$, we have got an~expression denoting~$W_q$ of star height~$q$.
\end{proof}

\begin{example}
    Let us have language~$W_3$ recognised by an automaton~${\mathcal{R}(8)}$. In~\autoref*{fig:automaton_R8_state_removal_steps} we show how the~state removal algorithm finds the~expression of star height~$3$ that denotes~$W_3$. For the expression at the end of the algorithm we can see the equivalence:
    \[
        {(Z_2 + {(X_2 + Y_2)}Z_2^*{(X_2 + Y_2)})}^* = {(X_3 + Y_3 + Z_3)}^* = W_3.
    \]
\end{example}

\begin{figure}%
    \centering
    \hspace{-20pt}%
    \subfloat[][]{\label{fig:automaton_R8_state_removal_steps-a}%
        \input{./figures/automaton_R8_removed1}}%
    \hspace{70pt}%
    \subfloat[][]{\label{fig:automaton_R8_state_removal_steps-b}%
        \begin{tikzpicture}[my_automaton]
    \tikzstyle{every state}=[my_state]
    \tikzstyle{initial}=[my_initial_state]
    \tikzstyle{accepting}=[my_accepting_state]
    %\draw[help lines] (-1.5,1.5) grid (1.5,-1.5);
    \node[state, initial, accepting] (0) at (180:1.5) {};
    \node[state] (2) at (90:1.5) {};
    \node[state] (4) at (0:1.5) {};
    \node[state] (5) at (-45:1.5) {};
    \node[state] (6) at (-90:1.5) {};
    \node[state] (7) at (-135:1.5) {};

    \path
        (0) edge [loop above, left, rotate=20] node {$ab$} ()
        (0) edge [above, bend left = 20] node [above] {$a^2$} (2)
        (2) edge [above, bend left = 20] node [near start, below=+1mm] {$b^2$} (0)
        (2) edge [loop above] node [right=+1mm] {$ab+ba$} ()
        (2) edge [above, bend left = 20] node [above right=-1mm] {$a^2$} (4)
        (4) edge [above, bend left = 20] node [below left=-1.5mm] {$b^2$} (2)
        (4) edge [loop right] node {$ba$} ()
        (4) edge [above, bend left = 20] node [right] {$a$} (5)
        (5) edge [above, bend left = 20] node [above left=-1mm] {$b$} (4)
        (5) edge [above, bend left = 20] node [below] {$a$} (6)
        (6) edge [above, bend left = 20] node [above] {$b$} (5)
        (6) edge [above, bend left = 20] node [below] {$a$} (7)
        (7) edge [above, bend left = 20] node [above] {$b$} (6)
        (7) edge [above, bend left = 20] node [left] {$a$} (0)
        (0) edge [above, bend left = 20] node [above right=-1mm] {$b$} (7)
    ;
\end{tikzpicture}}\\
    \subfloat[][]{\label{fig:automaton_R8_state_removal_steps-c}%
        \begin{tikzpicture}[my_automaton]
    \tikzstyle{every state}=[my_state]
    \tikzstyle{initial}=[my_initial_state]
    \tikzstyle{accepting}=[my_accepting_state]
    %\draw[help lines] (-1.5,1.5) grid (1.5,-1.5);
    \node[state, initial, accepting] (0) at (180:1.5) {};
    \node[state] (2) at (90:1.5) {};
    \node[state] (4) at (0:1.5) {};
    \node[state] (6) at (-90:1.5) {};
    \node[state] (7) at (-135:1.5) {};

    \path
        (0) edge [loop above, left, rotate=20] node {$ab$} ()
        (0) edge [above, bend left] node [above] {$a^2$} (2)
        (2) edge [above, bend left] node [near start, below=+1mm] {$b^2$} (0)
        (2) edge [loop above] node [right=+1mm] {$ab+ba$} ()
        (2) edge [above, bend left] node [above right=-1mm] {$a^2$} (4)
        (4) edge [above, bend left] node [below left=-1.5mm] {$b^2$} (2)
        (4) edge [loop right] node {$ab+ba$} ()
        (4) edge [above, bend left] node [right] {$a^2$} (6)
        (6) edge [above, bend left] node [above] {$b^2$} (4)
        (6) edge [loop below] node [right] {$ba$} ()
        (6) edge [above, bend left] node [below] {$a$} (7)
        (7) edge [above, bend left] node [above] {$b$} (6)
        (7) edge [above, bend left] node [left] {$a$} (0)
        (0) edge [above, bend left] node [above right=-1mm] {$b$} (7)
    ;
\end{tikzpicture}}%
    \hspace{8pt}%
    \subfloat[][]{\label{fig:automaton_R8_state_removal_steps-d}%
        \begin{tikzpicture}[my_automaton]
    \tikzstyle{every state}=[my_state]
    \tikzstyle{initial}=[my_initial_state]
    \tikzstyle{accepting}=[my_accepting_state]
    %\draw[help lines] (-1.5,1.5) grid (1.5,-1.5);
    \node[state, initial, accepting] (0) at (180:1.5) {};
    \node[state] (2) at (90:1.5) {};
    \node[state] (4) at (0:1.5) {};
    \node[state] (6) at (-90:1.5) {};

    \path
        (0) edge [loop above, left, rotate=20] node {$ab+ba$} ()
        (0) edge [above, bend left = 20] node [above] {$a^2$} (2)
        (2) edge [above, bend left = 20] node [near start, below=+1mm] {$b^2$} (0)
        (2) edge [loop above] node [right=+1mm] {$ab+ba$} ()
        (2) edge [above, bend left = 20] node [above right=-1mm] {$a^2$} (4)
        (4) edge [above, bend left = 20] node [below left=-1.5mm] {$b^2$} (2)
        (4) edge [loop right] node {$ab+ba$} ()
        (4) edge [above, bend left = 20] node [right] {$a^2$} (6)
        (6) edge [above, bend left = 20] node [above] {$b^2$} (4)
        (6) edge [loop below] node [right] {$ab+ba$} ()
        (6) edge [above, bend left = 20] node [below left=-1.5mm] {$a^2$} (0)
        (0) edge [above, bend left = 20] node [above right=-1.5mm] {$b^2$} (6)
    ;
\end{tikzpicture}}\\
    \subfloat[][]{\label{fig:automaton_R8_state_removal_steps-e}%
        \begin{tikzpicture}[my_automaton]
    \tikzstyle{every state}=[my_state]
    \tikzstyle{initial}=[my_initial_state]
    \tikzstyle{accepting}=[my_accepting_state]
    %\draw[help lines] (-1.5,1.5) grid (1.5,-1.5);
    \node[state, initial, initial above, accepting] (0) at (180:1.5) {};
    \node[state] (4) at (0:1.5) {};
    \node[state] (6) at (-90:1.5) {};

    \path
        (0) edge [loop left] node {$ab+ba+a^2{(ab+ba)}^*b^2$} ()
        (0) edge [above, bend left = 70] node [above] {$X_2$} (4)
        (4) edge [above, bend right = 50] node [below] {$Y_2$} (0)
        (4) edge [loop right] node {$ab+ba+b^2{(ab+ba)}^*a^2$} ()
        (4) edge [above, bend left = 20] node [right] {$a^2$} (6)
        (6) edge [above, bend left = 20] node [above] {$b^2$} (4)
        (6) edge [loop below] node [right] {$ab+ba$} ()
        (6) edge [above, bend left = 20] node [below left=-1mm] {$a^2$} (0)
        (0) edge [above, bend left = 20] node [above right=-1mm] {$b^2$} (6)
    ;
\end{tikzpicture}}\\
    \subfloat[][]{\label{fig:automaton_R8_state_removal_steps-f}%
        \begin{tikzpicture}[my_automaton]
    \tikzstyle{every state}=[my_state]
    \tikzstyle{initial}=[my_initial_state]
    \tikzstyle{accepting}=[my_accepting_state]
    %\draw[help lines] (-1.5,1.5) grid (1.5,-1.5);
    \node[state, initial, initial above, accepting] (0) at (180:1.5) {};
    \node[state] (4) at (0:1.5) {};

    \path
        (0) edge [loop left] node {$Z_2$} ()
        (0) edge [bend left] node [above] {$X_2 + Y_2$} (4)
        (4) edge [bend left] node [below] {$X_2 + Y_2$} (0)
        (4) edge [loop right] node {$Z_2$} ()
    ;
\end{tikzpicture}}%
    \hspace{40pt}%
    \subfloat[][]{\label{fig:automaton_R8_state_removal_steps-g}%
        \begin{tikzpicture}[->,>=latex,semithick]
    \tikzstyle{every state}=[circle, very thick, minimum size=3pt]
    \tikzstyle{initial}=[initial by arrow, initial text=]
    \tikzstyle{accepting}=[accepting by arrow]
    %\draw[help lines] (-1.5,1.5) grid (1.5,-1.5);
    \node[state, initial, initial above, accepting below] (0) at (180:1.5) {};

    \path
        (0) edge [loop right] node {$Z_2 + {(X_2 + Y_2)}Z_2^*{(X_2 + Y_2)}$} ()
    ;
\end{tikzpicture}}%
    \caption{First through seventh steps of the~state removal algorithm with order~$\omega$}\label{fig:automaton_R8_state_removal_steps}%
\end{figure}

\begin{defn}
    For~each integer~$n$ we define the~sequence $w_{0,n}, w_{1,n}, \dotsc , w_{q-1,n}$ of $q$~words in~$W_q$ recursively:
    \begin{alignat*}{2}
        w_{0,n} &= ab,\\
        w_{1,n} &= a^2{(ab)}^{n}b^2{(ab)}^{n},\\
                &\; \vdots \\
        w_{k,n} &= a^{2^k}{(w_{k-1,n})}^{n}b^{2^k}{(w_{k-1,n})}^{n},\\
                &\; \vdots \\
        w_{q-1,n} &= a^{2^{q-1}}{(w_{q-2,n})}^{n}b^{2^{q-1}}{(w_{q-2,n})}^{n}.\\
    \end{alignat*}
    We call these words \emph{witness words}.
\end{defn}

\begin{lemma}\label{lm:witness_words_inequalities}
    Any left factor $u$ and right factor $v$ of ${(w_{k,n})}^n$ satisfy the equations:
    \begin{equation}
        0 \leq |u|_a - |u|_b \leq 2^{k+1}-1 \; , \qquad 0 \leq |v|_b - |v|_a \leq 2^{k+1}-1.
    \end{equation}
\end{lemma}

\begin{proof}
    $|w_{k,n}|_a = |w_{k,n}|_b$, therefore any whole word $w_{k,n}$ that is a factor of either $u$ or $v$ does not affect the inequalities we are proving. We only need to examine the differences in letters of the left (resp.~right) factor of the word $w_{k,n}$. We proceed by indunction on~$k$. Trivially the inequalities hold for $w_{0,n}$.
    \[
        w_{k,n} = a^{2^k}{(w_{k-1,n})}^{n}b^{2^k}{(w_{k-1,n})}^{n}
    \] begins with $2^k$ $a$'s and from the induction hypothesis, we know that $w_{k-1,n}$ can have at most $2^k-1$ more $a$'s than $b$'s. That proves the case of the left factor.
    Similarly for the case of the right factor $v$. From the induction hypothesis we know that the right factor of $w_{k-1,n}$ can have at most $2^k-1$ more $b$'s than $a$'s. Therefore if $b^{2^k}{(w_{k-1,n})}^{n}$ is a right factor of $v$, the right factor of ${(w_{k-1,n})}^{n}$ is the left factor of $v$. We can see that $|v|_b - |v|_a \leq 2^k + 2^k -1$.
\end{proof}

Next we use the witness words to define a property of languages.

\begin{defn}
    We say that language $L$ \emph{satisfies property $(P_k)$} if there exists an infinite number of values of $n$ such that ${(w_{k,n})}^n$ is a~factor of at~least one word in~L.
\end{defn}

Note that if $L$ satisfies $(P_k)$, it also satisfies $(P_l)$ for $l \leq k$ since ${(w_{k-1,n})}^n$ is a factor of ${(w_{k,n})}^n$.

\begin{proof}[Proof of \autoref*{thm:main}]
    Lemma~\ref*{lm:expression_existence} gives us an~expression with star height~$q$ denoting the~language~$W_q$. Now we show that the~height has to be at~least~$q$.\\
    By $\mathfrak{W}_k$ we denote a family of languages $L$ that satisfy the following conditions:
    \begin{enumerate}
        \item[(i)] $L \subseteq W_q$,
        \item[(ii)] $L$ satisfies $(P_k)$,
        \item[(iii)] $L$ has a minimum star height of any language satisfying the first two conditions.
    \end{enumerate}
    Let $h_k$ be the common value of the star height of the languages in $\mathfrak{W}_k$, then we have
    \[
        0 < h_0 \leq h_1 \leq \dotsb \leq h_{q-1} \leq q.
    \]
    Language in $\mathfrak{W}_0$ is necessarily infinite, therefore $0 < h_0$. Since $(P_k)$ implies $(P_{k-1})$, $h_{k-1} \leq h_k$. $W_q$ obviously satisfies (i), but it also satisfies $(P_{q-1})$ because for each $n$, $|{(w_{q-1,n})}^n|_a = |{(w_{q-1,n})}^n|_b$, and because in Lemma~\ref*{lm:expression_existence} we have found an expression of star height $q$, to satisfy the condition (iii), $h_{q-1} \leq q$.

    To prove the theorem it is enough to show that
    \[
        \forall \, k \in \{1,\dotsc,q-1\}: h_{k-1} < h_k.
    \]
    Let $L \in \mathfrak{W}_k$ with star height $h_k$. Due to Lemma~\ref*{lm:distributivity} $L$ can be written as a finite union of languages $E_j$, each denoted by a expression of the form:
    \[
        \mathsf{E}_j = u_0{(\mathsf{H}_1)}^*u_1 \dotsm u_{m-1}{(\mathsf{H}_m)}^*u_m.
    \]
    Each expression $\mathsf{H}_i$ denotes a~language~$H_i$ with star height less then or equal to $h_k-1$. At least one of the languages $E_j$ has to satisfy $(P_k)$ since $L$ satisfies it and there in only a finite number of languages $E_j$ in the union equal to $L$. It is therefore safe to assume that $L$ is denoted by an expression of the same form as $E_j$.

    Each of the $H_i$ must be contained in $W_q$. For contradiction, let us have word $g \in H_i$, that $|g|_a \not\equiv |g|_b \pmod{2^q}$. Then it must be true that both
    \[
        u_0 u_1 \dotsm u_m \in L \qquad \text{and} \qquad u_0 u_1 \dotsm u_{i-1} g u_{i+1} \dotsm u_m \in L,
    \]
    which contradicts the fact that $\forall f \in L: |f|_a \equiv |f|_b \pmod{2^q}$.

    Since $L$ is of a minimal star height to satisfy $(P_k)$, none of the $H_i$ satisfies $(P_k)$. Now we need to show that some $H_i$ satisfies $(P_{k-1})$. In fact we will have $h_{k-1} \leq h_k - 1$.

    $L$ satisfies $(P_k)$, so for arbitrarily large $n$, words ${(w_{k,n})}^n$ are factors of $L$. \X{Meaning that $\forall n \; \exists f \in L: {(w_{k,n})}^n$ is a factor of $f$? Hopefully.} Lemma~\ref*{lm:block_star_lemma} shows, that we can find $N$ large enough, that ${(w_{k,N})}^N$ will be a factor of $f \in L$, which still gives us infinitely many $n' \geq N$, that ${(w_{k,n'})}^{n'}$ is a factor some word, and for each of those $n'$ we have infinitely many $l$'s, that ${(w_{k,n})}^l$ is a factor of $L$\X{?}, again due to Lemma~\ref*{lm:block_star_lemma}. Since $m$ is finite and fixed for $L$, there has to be infinitely many $n$'s, that ${(w_{k,n})}^n$ is a factor a word $r_n \in H_i^*$, for some $i$, $1 \leq i \leq m$, therefore $H_i^*$ satisfies $(P_k)$. We will show that for these $n$'s, ${(w_{k-1,n})}^n$ is a factor of a word in $H_i$, meaning $H_i$ satisfies $(P_{k-1})$.

    We write $r_n$ as a factorization $r_n = g_0 g_1 \dotsm g_l$, where $g_j \in H_i$, for $0 \leq j \leq l$. If $w_{k,n}$, from ${(w_{k,n})}^n$, is a factor of some $g_j$, the condition is satisfied, since ${(w_{k-1,n})}^n$ is a factor of $w_{k,n}$. Otherwise we will examine how a factor ${(w_{k,n})}^2$ is covered by the factorization of $r_n$. Written explicitly, we have
    \[
        {(w_{k,n})}^2 = a^{2^k}{(w_{k-1,n})}^{n}b^{2^k}{(w_{k-1,n})}^{n}a^{2^k}{(w_{k-1,n})}^{n}b^{2^k}{(w_{k-1,n})}^{n}.
    \]
    Let us consider $g_j$, that the leftmost factor $b^{2^k}$ is covered by it, at least partially. There are two possibilities:
    \begin{enumerate}
        \item[(i)] $b^{2^k}$ is a factor of $g_j$,
        \item[(ii)] a left factor of $b^{2^k}$ is a right factor of $g_j$.
    \end{enumerate}

    In the case (i), we have $g_j = v b^{2^k} u$. If $v$ covers the factor ${(w_{k-1,n})}^{n}$ to the left of $b^{2^k}$, or $u$ covers the factor ${(w_{k-1,n})}^{n}$ to the right of $b^{2^k}$, the condition is satisfied. If not, $u$ is a left factor of ${(w_{k-1,n})}^{n}$ and $v$ is a right factor. Set
    \[
        x = |g_j|_b - |g_j|_a, \quad y = |u|_a - |u|_b, \quad \text{ and } \quad z = |v|_b - |v|_a.
    \]
    Hence
    \[
        x = 2^k - (y - z).
    \]
    We see that $1 - 2^k \leq y - z \leq 2^k - 1$, from Lemma~\ref*{lm:witness_words_inequalities}. That gives us
    \[
        0 < x < 2^{k+1} \leq 2^q.
    \]
    which contradicts the fact that $g_j \in H_i \subseteq W_q$.

    In the case (ii) we have $g_j = v b^{r}, \: 0 < r < 2^k$. If ${(w_{k-1,n})}^{n}$ is a factor of $v$, the condition is satisfied. Otherwise $v$ is a right factor of ${(w_{k-1,n})}^{n}$ and, similarly as above, we set
    \[
        x = |g_j|_b - |g_j|_a, \quad \text{ and } \quad z = |v|_b - |v|_a.
    \]
    Hence
    \[
        x = r + z.
    \]
    Therefore, due to Lemma~\ref*{lm:witness_words_inequalities},
    \[
        r \leq x \leq r + 2^k - 1 < 2^{k+1} < 2^q.
    \]
    which produces the same contradiction as the case (i).
\end{proof}