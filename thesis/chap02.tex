\chapter{Eggan's question}

In this chapter we present an answer to the question whether there are rational languages with arbitrarily large star heights. The first example of such languages is given by Eggan~\cite{Eggan63}. Eggan's example uses an~alphabet of size $2^n - 1$ for the language with star height $n$. He therefore asked whether there are some examples of languages over binary alphabets. Answer to that question, which is affirmative, was provided shortly after by Dejean and Schützenberger~\cite{DejeanSchutzenberger66}. Their answer is also reformulated by Sakarovitch~\cite{Sakarovitch09}, who\X{/which?} provided the~basis of this work.

The family of languages provided by Dejean and Schützenberger is the following:

\begin{defn}
    Let $q$ be a~non-negative integer. Language over $\{a, b\}$ is called $W_q$, if for each word $f$ in $W_q$:
    \[
        |f|_a \equiv |f|_b \pmod{2^q}.
    \]
\end{defn}

\begin{thm}\label{thm:main}
    The language $W_q$ has star height~$q$.
\end{thm}

The proof has three parts. First, we find an automaton that recognises $W_q$ in Lemma~\ref*{lm:W_q_recognised_by_R2q}. Second, we infer a~rational expression of star height~$q$, denoting $W_q$, in Lemma~\ref*{lm:expression_existence}. And lastly, we show that the star height of the language has to be at~least $q$.

\begin{lemma}\label{lm:W_q_recognised_by_R2q}
    Language $W_q$ is recognised by an automaton~$\aut{R}(2^q)$.
\end{lemma}

\begin{proof}
    We show that each word in $W_q$ is accepted by a~successful computation in $\aut{R}(2^q)$. Let $f$ be a~word in $W_q$. Since $f$ has even length, let $k$ be such that $2k = |f|$. Since $f \in W_q$, $|f|_a - |f|_b = n 2^q$ for some integer $n$.

    For $n=0$, $|f|_a = |f|_b$. We use induction on $k$ to show that for every factorisation $f = uvw$, if $f$ is accepted by $\aut{R}(2^q)$, $uavbw$ and $ubvaw$ are also. For $k = 0$, empty word is accepted since the initial state is also final and both computations $0 \tr{a} 1 \tr{b} 0$, and $0 \tr{b} 2^q-1 \tr{a} 0$ are successful in $\aut{R}(2^q)$. Given the induction hypothesis for some $k - 1$, every word $f = uvw$ of length $2k - 2$ such that $|f|_a = |f|_b$ is accepted by computation
    \[
        0 \tr{\re{E}} i \tr{\re{F}} j \tr{\re{G}} 0 \, ,
    \]
    where $u \in L[\re{E}], v \in L[\re{F}]$, and $v \in L[\re{G}]$. We need to show that computations
    \begin{align*}
        0 &\tr{\re{E}} i \tr{a} i+1 \tr{\re{F}} j+1 \tr{b} j \tr{\re{G}} 0 \, , \quad \text{and} \\
        0 &\tr{\re{E}} i \tr{b} i-1 \tr{\re{F}} j-1 \tr{a} j \tr{\re{G}} 0 \, ,
    \end{align*}
    are successful. For any two states $r$ and $t$ of $\aut{R}(2^q)$, the computation $r \tr{\re{H}} t$ can be written as a sequence of transitions:
    \[
        r \tr{\re{X}_1} s_1 \tr{\re{X}_2} \dotsb \tr{\re{X}_{m-1}} s_{m-1} \tr{\re{X}_m} t \, ,
    \]
    where each $\re{X}_l$ is either a letter $a$ or $b$. Due to Lemma~\ref*{lm:R_n_computation_existence}, computations
    \begin{align*}
        r + 1 &\tr{\re{X}_1} s_1 + 1 \tr{\re{X}_2} \dotsb \tr{\re{X}_{m-1}} s_{m-1} + 1 \tr{\re{X}_m} t + 1 \, , \quad \text{and} \\
        r - 1 &\tr{\re{X}_1} s_1 - 1 \tr{\re{X}_2} \dotsb \tr{\re{X}_{m-1}} s_{m-1} - 1 \tr{\re{X}_m} t - 1 \\
    \intertext{are also in $\aut{R}(2^q)$, and so are}
        r \tr{a} r + 1 &\tr{\re{X}_1} s_1 + 1 \tr{\re{X}_2} \dotsb \tr{\re{X}_{m-1}} s_{m-1} + 1 \tr{\re{X}_m} t + 1 \tr{b} t \, , \quad \text{and} \\
        r \tr{b} r - 1 &\tr{\re{X}_1} s_1 - 1 \tr{\re{X}_2} \dotsb \tr{\re{X}_{m-1}} s_{m-1} - 1 \tr{\re{X}_m} t - 1 \tr{a} t \, .
    \end{align*}
    If we take computation $i \tr{\re{F}} j$ in place of $r \tr{\re{H}} t$, the assertion is proved.

    For $n \neq 0$, $|f|_a - |f|_b = n 2^q$. Let $g$ be a word such that $|g|_a = |g|_b$, therefore $g$ is accepted by $\aut{R}(2^q)$. Computation accepting $g$ can be written as
    \[
        0 \tr{\re{X}_1} s_1 \tr{\re{X}_2} \dotsb \tr{\re{X}_{m-1}} s_{m-1} \tr{\re{X}_m} 0 \, ,
    \]
    where each $\re{X}_l$ is either a letter $a$ or $b$. Let $\re{Y}_1, \dotsc, \re{Y}_{m+1}$ be expressions such that $f$ is in $L[\re{Y}_1 \re{X}_1 \dotsm \re{Y}_m \re{X}_m \re{Y}_{m+1}]$. Let $e_h$ be in $L[\re{Y}_h]$ for $1 \leq h \leq m+1$.
    \[
        n 2^q = |f|_a - |f|_b = |g|_a + \sum_{h=1}^{m+1} |e_h|_a - |g|_b - \sum_{h=1}^{m+1} |e_h|_b = \sum_{h=1}^{m+1} (|e_h|_a - |e_h|_b) \, .
    \]
    By $m + 1$ iterations of Lemma~\ref*{lm:R_n_computation_existence}, it follows that computation
    \begin{multline*}
        0 \tr{\re{Y}_1 \re{X}_1} p_1 + (|e_1|_a - |e_1|_b) \tr{\re{Y}_2 \re{X}_2} p_2 + \sum_{h=1}^{2} (|e_h|_a - |e_h|_b) \tr{\re{Y}_3 \re{X}_3} \dotsb \\
        \dotsb \tr{\re{Y}_{m-1} \re{X}_{m-1}} p_{m-1} + \sum_{h=1}^{m-1} (|e_h|_a - |e_h|_b) \tr{\re{Y}_m \re{X}_m \re{Y}_{m+1}} \sum_{h=1}^{m+1} (|e_h|_a - |e_h|_b) \, .
    \end{multline*}
    is in $\aut{R}(2^q)$. It is successful computation in $\aut{R}(2^q)$ because
    \[
        n 2^q = \sum_{h=1}^{m+1} (|e_h|_a - |e_h|_b) \equiv 0 \pmod{2^q}.
    \]
\end{proof}

\begin{lemma}\label{lm:expression_existence}
    Language $W_q$ is denoted by a rational expression of star height $q$.
\end{lemma}

\begin{proof}
    Due to Lemma~\ref*{lm:W_q_recognised_by_R2q}, language $W_q$ is recognised by an~automaton $\aut{R}(2^q)$. By iterative application of the state removal algorithm, we will obtain an~automaton with only one state. First, we inductively define the following rational expressions:
    \begin{align*}
        \re{X}_0 &= a \, , & \re{Y}_0 &= b \, , & &\text{ and } & \re{Z}_0 &= \re{0} \, , \\
        \re{X}_1 &= a^2 \, , & \re{Y}_1 &= b^2 \, , & &\text{ and } & \re{Z}_1 &= ab+ba \, , \\
        &\; \vdots & &\; \vdots & & & &\; \vdots \\
        \re{X}_{n+1} &= \re{X}_n \re{Z}_n^* \re{X}_n \, , & \re{Y}_{n+1} &= \re{Y}_n \re{Z}_n^* \re{Y}_n \, , & &\text{ and } & \re{Z}_{n+1} &= \re{Z}_n + \re{X}_n \re{Z}_n^* \re{Y}_n + \re{Y}_n \re{Z}_n^* \re{X}_n \, .
    \end{align*}
    Second, we define automaton $\aut{A}_k$ as $\aut{A}_k = \langle \Z_{2^{q-k}}, A, E, \{0\}, \{0\} \rangle$ such that any state $p$ in $\aut{A}_k$ is source of exactly three transitions in $\aut{A}_k$:
    \[
        p \tr{\re{X}_k} p+1 \, , \quad p \tr{\re{Y}_k} p-1 \, , \quad \text{and} \quad p \tr{\re{Z}_k} p \, .
    \]
    By induction on $k$, we show that $W_q$, accepted by $\aut{R}(2^q)$, is also accepted by $\aut{A}_k$.

    Let $k = 0$. Note that for each state $p$ in $\aut{A}_0$, transitions
    \[
        p \tr{\re{X}_0} p+1 \, , \quad p \tr{\re{Y}_0} p-1 \, , \quad \text{and} \quad p \tr{\re{Z}_0} p
    \]
    are in $\aut{R}(2^q)$. Similarly, for any state $r$ in $\aut{R}(2^q)$, transitions
    \[
        r \tr{a} r+1 \, , \quad r \tr{b} r-1
    \]
    are in $\aut{A}_0$. Therefore $\aut{R}(2^q)$ is equivalent to $\aut{A}_0$.

    Given the induction hypothesis for some $k - 1$, $W_q$ is accepted by $\aut{A}_{k-1}$. By $2^{q-k}$ iterations of the state removal algorithm an automaton with $2^{q-k}$ states is obtained. The set of states of $\aut{A}_{k-1}$ is $\Z_{2^{q-k+1}}$. Consecutively, in the increasing order, we remove all the odd states of $\aut{A}_{k-1}$, that is, states $1, 3, 5, \dotsc, 2^{q-k+1}-1$. Let $m$ be the removed odd state.

    For $m = 1$, the transitions that have $1$ as a source, destination, or both are:
    \[
        1 \tr{\re{X}_{k-1}} 2 \, , \quad 1 \tr{\re{Y}_{k-1}} 0 \, , \quad 0 \tr{\re{X}_{k-1}} 1 \, , \quad 2 \tr{\re{Y}_{k-1}} 1 \, , \quad \text{and} \quad 1 \tr{\re{Z}_{k-1}} 1 \, .
    \]
    Those transitions are removed, in accordance with the state removal algorithm. For each pair of states, $(0,0), (0,2), (2,0), (2,2)$, these transitions are replaced:
    \begin{align*}
        &0 \tr{\re{Z}_{k-1}} 0 & &\text{replaced by} & &0 \tr{\re{Z}_{k-1} \: + \: \re{X}_{k-1}\re{Z}_{k-1}^*\re{Y}_{k-1}} 0 \, , \\
        &0 \tr{\re{E}} 2 & &\text{replaced by} & &0 \tr{\re{E} \: + \: \re{X}_{k-1}\re{Z}_{k-1}^*\re{X}_{k-1}} 2 \, , \\
        &2 \tr{\re{E}} 0 & &\text{replaced by} & &2 \tr{\re{E} \: + \: \re{Y}_{k-1}\re{Z}_{k-1}^*\re{Y}_{k-1}} 0 \, , \\
        &2 \tr{\re{Z}_{k-1}} 2 & &\text{replaced by} & &2 \tr{\re{Z}_{k-1} \: + \: \re{Y}_{k-1}\re{Z}_{k-1}^*\re{X}_{k-1}} 2 \, .
    \end{align*}

    For $3 \leq m \leq 2^{q-k+1}-3$, the states $1, 3, \dotsc, m-2$ are already removed. The transitions that have $m$ as a source, destination, or both are:
    \begin{multline*}
        m \tr{\re{X}_{k-1}} m+1 \, , \quad m \tr{\re{Y}_{k-1}} m-1 \, , \quad m-1 \tr{\re{X}_{k-1}} m \, , \\
        m+1 \tr{\re{Y}_{k-1}} m \, , \quad \text{and} \quad m \tr{\re{Z}_{k-1}} m \, .
    \end{multline*}
    As in the case $m = 1$, those transitions are removed. For each pair of states, $(m-1,m-1), (m-1,m+1), (m+1,m-1), (m+1,m+1)$, these transitions are replaced:
    \begin{align*}
        &m \! - \! 1 \tr{\re{Z}_{k-1} \: + \: \re{Y}_{k-1}\re{Z}_{k-1}^*\re{X}_{k-1}} m \! - \! 1 & &\text{replaced by} & &m \! - \! 1 \tr{\re{Z}_k} m-1 \, , \\
    \intertext{since $\re{Z}_k = (\re{Z}_{k-1} + \re{Y}_{k-1}\re{Z}_{k-1}^*\re{X}_{k-1}) + \re{X}_{k-1}\re{Z}_{k-1}^*\re{Y}_{k-1} \, $,}
        &m \! - \! 1 \tr{\re{E}} m+1 & &\text{replaced by} & &m \! - \! 1 \tr{\re{E} \: + \: \re{X}_{k-1}\re{Z}_{k-1}^*\re{X}_{k-1}} m+1 \, , \\
        &m \! + \! 1 \tr{\re{E}} m-1 & &\text{replaced by} & &m \! + \! 1 \tr{\re{E} \: + \: \re{Y}_{k-1}\re{Z}_{k-1}^*\re{Y}_{k-1}} m-1 \, , \\
        &m \! + \! 1 \tr{\re{Z}_{k-1}} m+1 & &\text{replaced by} & &m \! + \! 1 \tr{\re{Z}_{k-1} \: + \: \re{Y}_{k-1}\re{Z}_{k-1}^*\re{X}_{k-1}} m \! + \! 1.
    \end{align*}

    Lastly for $m = 2^{q-k+1}-1$,  the states $1, 3, \dotsc, 2^{q-k+1}-3$ are already removed. Since $m = 2^{q-k+1}-1$, the state $m+1$ is $0 \equiv 2^{q-k+1} \pmod{2^{q-k+1}}$. The transitions that have $m$ as a source, destination, or both are:
    \begin{multline*}
        m \tr{\re{X}_{k-1}} 0 \, , \quad m \tr{\re{Y}_{k-1}} m-1 \, , \quad m-1 \tr{\re{X}_{k-1}} m \, , \\
        0 \tr{\re{Y}_{k-1}} m \, , \quad \text{and} \quad m \tr{\re{Z}_{k-1}} m \, .
    \end{multline*}
    As for every $m \leq 2^{q-k+1}-3$, these transitions are removed. For each pair of states, $(m-1,m-1), (m-1,0), (0,m-1), (0,0)$, these transitions are replaced:
    \begin{align*}
        m \! - \! 1 &\tr{\re{Z}_{k-1} \: + \: \re{Y}_{k-1}\re{Z}_{k-1}^*\re{X}_{k-1}} m \! - \! 1 & &\text{replaced by} & m \! - \! 1 &\tr{\re{Z}_k} m-1 \, , \\
        m \! - \! 1 &\tr{\re{E}} 0 & &\text{replaced by} & m \! - \! 1 &\tr{\re{E} \: + \: \re{X}_{k-1}\re{Z}_{k-1}^*\re{X}_{k-1}} 0 \, , \\
        0 &\tr{\re{E}} m-1 & &\text{replaced by} & 0 &\tr{\re{E} \: + \: \re{Y}_{k-1}\re{Z}_{k-1}^*\re{Y}_{k-1}} m \! - \! 1 \, , \\
        0 &\tr{\re{Z}_{k-1} \: + \: \re{X}_{k-1}\re{Z}_{k-1}^*\re{Y}_{k-1}} 0 & &\text{replaced by} & 0 &\tr{\re{Z}_k} 0 \, .
    \end{align*}

    Thus, after $2^{q-k}$ odd states removed, we have automaton with $2^{q-k}$ states, all non-negative even numbers $\leq 2^{q-k+1}-2$, that accepts $W_q$. To complete the inductive step, create automaton $\aut{A}_k$, we `rename' each state $2t$ as $t$, $t \in \{ 0, 1, 2, \dotsc, 2^{q-k}-1 \}$, meaning we replace each state $2t$ with $t$, each transition $2t \tr{\re{F}} s$ with $t \tr{\re{F}} s$, and $s \tr{\re{F}} 2t$ with $s \tr{\re{F}} t$, for every state $s$.

    After $q$ steps of the induction on $k$ we cannot further reduce the states of $\aut{A}_q$ since it only has one state. This automaton has also only one transition, a loop on the state $0$, with label $\re{Z}_{q-1} + (\re{X}_{q-1} + \re{Y}_{q-1})\re{Z}_{q-1}^*(\re{X}_{q-1} + \re{Y}_{q-1})$. Note the following equivalence:
    \[
        \re{Z}_{q-1} + (\re{X}_{q-1} + \re{Y}_{q-1})\re{Z}_{q-1}^*(\re{X}_{q-1} + \re{Y}_{q-1}) \equiv \re{X}_q + \re{Y}_q + \re{Z}_q \, . \\
    \]
    Therefore
    \[
        W_q = L(\aut{A}_q) = L[{(\re{Z}_{q-1} + (\re{X}_{q-1} + \re{Y}_{q-1})\re{Z}_{q-1}^*(\re{X}_{q-1} + \re{Y}_{q-1}))}^*] = L[{(\re{X}_q \! + \! \re{Y}_q \! + \! \re{Z}_q)}^*].
    \]

    By induction on $i$, we show, that $\h{\re{X}_i} = \h{\re{Z}_i} = \h{\re{Z}_i} = i-1$. Let $i=1$. $\h{a^2} = \h{b^2} = \h{ab+ba} = 0$. Given the induction hypothesis for some $i - 1$,
    $\h{\re{Z}_{i-1}} = i-1$. Therefore
    \begin{align*}
        \h{\re{Z}_{i-1}^*} = i &= \h{\re{X}_{i-1} \re{Z}_{i-1}^* \re{X}_{i-1}} = \h{\re{X}_i} \\
        &= \h{\re{Y}_{i-1} \re{Z}_{i-1}^* \re{Y}_{i-1}} = \h{\re{Y}_i} \\
        &= \h{\re{Z}_{i-1} + \re{X}_{i-1} \re{Z}_{i-1}^* \re{Y}_{i-1} + \re{Y}_{i-1} \re{Z}_{i-1}^* \re{X}_{i-1}} = \h{\re{Z}_i} \, .
    \end{align*}
    We conclude that $\h{{(\re{X}_q + \re{Y}_q + \re{Z}_q)}^*} = q$, which proves the lemma.
\end{proof}

\begin{example}
    Language $W_3$ is recognised by $\aut{R}(8)$. In~\autoref*{fig:automaton_R8_state_removal_steps}, we show how the rational expression $\re{X}_3 + \re{Y}_3 + \re{Z}_3$ is derived. Then $W_3 = L[{(\re{X}_3 + \re{Y}_3 + \re{Z}_3)}^*]$.

    \begin{figure}[h]%
        \centerline{
            \hspace{-15mm}%
            \subfloat[Automaton~$\aut{R}(8)$]{%
                \begin{tikzpicture}[my_automaton]
    \tikzstyle{every state}=[my_state]
    \tikzstyle{initial}=[my_initial_state]
    \tikzstyle{accepting}=[my_accepting_state]
    %\draw[help lines] (-1.5,1.5) grid (1.5,-1.5);
    \node[state, initial, accepting] (0) at (180:1.5) {};
    \node[state] (1) at (135:1.5) {};
    \node[state] (2) at (90:1.5) {};
    \node[state] (3) at (45:1.5) {};
    \node[state] (4) at (0:1.5) {};
    \node[state] (5) at (-45:1.5) {};
    \node[state] (6) at (-90:1.5) {};
    \node[state] (7) at (-135:1.5) {};

    \path
        (0) edge [bend left] node [near start, above left=-1mm] {$a$} (1)
        (1) edge [bend left] node [right] {$b$} (0)
        (1) edge [bend left] node [above] {$a$} (2)
        (2) edge [bend left] node [below right=-1mm] {$b$} (1)
        (2) edge [bend left] node [above right=-1mm] {$a$} (3)
        (3) edge [bend left] node [near start, below left=-1mm] {$b$} (2)
        (3) edge [bend left] node [right] {$a$} (4)
        (4) edge [bend left] node [left] {$b$} (3)
        (4) edge [bend left] node [right] {$a$} (5)
        (5) edge [bend left] node [above left=-1mm] {$b$} (4)
        (5) edge [bend left] node [below] {$a$} (6)
        (6) edge [bend left] node [above] {$b$} (5)
        (6) edge [bend left] node [below] {$a$} (7)
        (7) edge [bend left] node [above] {$b$} (6)
        (7) edge [bend left] node [left] {$a$} (0)
        (0) edge [bend left] node [above right=-1mm] {$b$} (7)
    ;
\end{tikzpicture}}%
            \qquad
            \subfloat[Automaton~$\aut{A}_0$]{%
                \begin{tikzpicture}[my_automaton]
    \tikzstyle{every state}=[my_named_state]
    \tikzstyle{initial}=[my_initial_state]
    \tikzstyle{accepting}=[my_accepting_state]
    %\draw[help lines] (-1.8,1.8) grid (1.8,-1.8);
    \node[state, initial, accepting] (0) at (180:1.8) {0};
    \node[state] (1) at (135:1.8) {1};
    \node[state] (2) at (90:1.8) {2};
    \node[state] (3) at (45:1.8) {3};
    \node[state] (4) at (0:1.8) {4};
    \node[state] (5) at (-45:1.8) {5};
    \node[state] (6) at (-90:1.8) {6};
    \node[state] (7) at (-135:1.8) {7};

    \path
        (0) edge [loop right] node {$\re{0}$} ()
        (0) edge [bend left] node [near start, above left=-1mm] {$a$} (1)
        (1) edge [bend left] node [right] {$b$} (0)
        (1) edge [in=120, out=155, distance=0.7cm, loop, above] node {$\re{0}$} ()
        (1) edge [bend left] node [above] {$a$} (2)
        (2) edge [bend left] node [below right=-1mm] {$b$} (1)
        (2) edge [loop above] node [right=+0.5mm] {$\re{0}$} ()
        (2) edge [bend left] node [above right=-1mm] {$a$} (3)
        (3) edge [bend left] node [near start, below left=-1mm] {$b$} (2)
        (3) edge [in=30, out=60, distance=0.7cm, loop, above] node {$\re{0}$} ()
        (3) edge [bend left] node [right] {$a$} (4)
        (4) edge [bend left] node [left] {$b$} (3)
        (4) edge [loop left] node {$\re{0}$} ()
        (4) edge [bend left] node [right] {$a$} (5)
        (5) edge [bend left] node [left=-0.5mm] {$b$} (4)
        (5) edge [in=-60, out=-30, distance=0.7cm, loop, below] node {$\re{0}$} ()
        (5) edge [bend left] node [below] {$a$} (6)
        (6) edge [bend left] node [above] {$b$} (5)
        (6) edge [loop below] node [right] {$\re{0}$} ()
        (6) edge [bend left] node [below] {$a$} (7)
        (7) edge [bend left] node [above] {$b$} (6)
        (7) edge [in=-155, out=-120, distance=0.7cm, loop, below] node {$\re{0}$} ()
        (7) edge [bend left] node [left] {$a$} (0)
        (0) edge [bend left] node [near end, above right=-1.2mm] {$b$} (7)
    ;
\end{tikzpicture}}%
            \qquad
            \subfloat[]{%
                \begin{tikzpicture}[my_automaton]
    \tikzstyle{every state}=[my_named_state]
    \tikzstyle{initial}=[my_initial_state]
    \tikzstyle{accepting}=[my_accepting_state]
    %\draw[help lines] (-1.8,1.8) grid (1.8,-1.8);
    \node[state, initial, accepting] (0) at (180:1.8) {0};
    \node[state] (2) at (90:1.8) {2};
    \node[state] (3) at (45:1.8) {3};
    \node[state] (4) at (0:1.8) {4};
    \node[state] (5) at (-45:1.8) {5};
    \node[state] (6) at (-90:1.8) {6};
    \node[state] (7) at (-135:1.8) {7};

    \path
        (0) edge [loop right] node {$ab$} ()
        (0) edge [bend left] node [above] {$a^2$} (2)
        (2) edge [bend left=15] node [near start, below=+1mm] {$b^2$} (0)
        (2) edge [loop above] node [right=+1mm] {$ba$} ()
        (2) edge [bend left] node [above right=-1mm] {$a$} (3)
        (3) edge [bend left] node [near start, below left=-1mm] {$b$} (2)
        (3) edge [in=30, out=60, distance=0.7cm, loop, above] node {$\re{0}$} ()
        (3) edge [bend left] node [right] {$a$} (4)
        (4) edge [bend left] node [left] {$b$} (3)
        (4) edge [loop left] node {$\re{0}$} ()
        (4) edge [bend left] node [right] {$a$} (5)
        (5) edge [bend left] node [left=-0.5mm] {$b$} (4)
        (5) edge [in=-60, out=-30, distance=0.7cm, loop, below] node {$\re{0}$} ()
        (5) edge [bend left] node [below] {$a$} (6)
        (6) edge [bend left] node [above] {$b$} (5)
        (6) edge [loop below] node [right] {$\re{0}$} ()
        (6) edge [bend left] node [below] {$a$} (7)
        (7) edge [bend left] node [above] {$b$} (6)
        (7) edge [in=-155, out=-120, distance=0.7cm, loop, below] node {$\re{0}$} ()
        (7) edge [bend left] node [left] {$a$} (0)
        (0) edge [bend left] node [near end, above right=-1.2mm] {$b$} (7)
    ;
\end{tikzpicture}}
        }
        \vspace{5mm}
        \centerline{
            \hspace{-15mm}%
            \subfloat[]{%
                \begin{tikzpicture}[my_automaton]
    \tikzstyle{every state}=[my_named_state]
    \tikzstyle{initial}=[my_initial_state]
    \tikzstyle{accepting}=[my_accepting_state]
    %\draw[help lines] (-1.8,1.8) grid (1.8,-1.8);
    \node[state, initial, accepting] (0) at (180:1.8) {0};
    \node[state] (2) at (90:1.8) {2};
    \node[state] (4) at (0:1.8) {4};
    \node[state] (5) at (-45:1.8) {5};
    \node[state] (6) at (-90:1.8) {6};
    \node[state] (7) at (-135:1.8) {7};

    \path
        (0) edge [loop right] node {$ab$} ()
        (0) edge [bend left] node [above] {$a^2$} (2)
        (2) edge [bend left=15] node [near start, below=+1mm] {$b^2$} (0)
        (2) edge [loop above] node [right=+1mm] {$ab+ba$} ()
        (2) edge [bend left] node [above right=-1mm] {$a^2$} (4)
        (4) edge [bend left] node [below left=-1.5mm] {$b^2$} (2)
        (4) edge [loop left] node {$ba$} ()
        (4) edge [bend left] node [right] {$a$} (5)
        (5) edge [bend left] node [left=-0.5mm] {$b$} (4)
        (5) edge [in=-60, out=-30, distance=0.7cm, loop, below] node {$\re{0}$} ()
        (5) edge [bend left] node [below] {$a$} (6)
        (6) edge [bend left] node [above] {$b$} (5)
        (6) edge [loop below] node [right] {$\re{0}$} ()
        (6) edge [bend left] node [below] {$a$} (7)
        (7) edge [bend left] node [above] {$b$} (6)
        (7) edge [in=-155, out=-120, distance=0.7cm, loop, below] node {$\re{0}$} ()
        (7) edge [bend left] node [left] {$a$} (0)
        (0) edge [bend left] node [near end, above right=-1.2mm] {$b$} (7)
    ;
\end{tikzpicture}}%
            \quad
            \subfloat[]{%
                \begin{tikzpicture}[my_automaton]
    \tikzstyle{every state}=[my_named_state]
    \tikzstyle{initial}=[my_initial_state]
    \tikzstyle{accepting}=[my_accepting_state]
    %\draw[help lines] (-1.8,1.8) grid (1.8,-1.8);
    \node[state, initial, accepting] (0) at (180:1.8) {0};
    \node[state] (2) at (90:1.8) {2};
    \node[state] (4) at (0:1.8) {4};
    \node[state] (6) at (-90:1.8) {6};
    \node[state] (7) at (-135:1.8) {7};

    \path
        (0) edge [in=120, out=155, distance=0.7cm, loop, above] node {$ab$} ()
        (0) edge [bend left] node [above] {$a^2$} (2)
        (2) edge [bend left=15] node [near start, below=+1mm] {$b^2$} (0)
        (2) edge [loop above] node [right=+1mm] {$ab+ba$} ()
        (2) edge [bend left] node [above right=-1mm] {$a^2$} (4)
        (4) edge [bend left=15] node [near end, below left=-1.5mm] {$b^2$} (2)
        (4) edge [loop left] node {$ab+ba$} ()
        (4) edge [bend left] node [right] {$a^2$} (6)
        (6) edge [bend left=15] node [near start, above] {$b^2$} (4)
        (6) edge [loop below] node [right] {$ba$} ()
        (6) edge [bend left] node [below] {$a$} (7)
        (7) edge [bend left] node [above] {$b$} (6)
        (7) edge [in=-155, out=-120, distance=0.7cm, loop, below] node {$\re{0}$} ()
        (7) edge [bend left] node [near start, below left=-1mm] {$a$} (0)
        (0) edge [bend left] node [near end, above right=-1.2mm] {$b$} (7)
    ;
\end{tikzpicture}}
            \quad
            \subfloat[Automaton~$\aut{A}_1$]{%
                \begin{tikzpicture}[my_automaton]
    \tikzstyle{every state}=[my_named_state]
    \tikzstyle{initial}=[my_initial_state]
    \tikzstyle{accepting}=[my_accepting_state]
    %\draw[help lines] (-1.8,1.8) grid (1.8,-1.8);
    \node[state, initial, accepting] (0) at (180:1.8) {0};
    \node[state] (1) at (90:1.8) {1};
    \node[state] (2) at (0:1.8) {2};
    \node[state] (3) at (-90:1.8) {3};

    \path
        (0) edge [in=120, out=155, distance=0.7cm, loop, above] node {$ab+ba$} ()
        (0) edge [bend left] node [above] {$a^2$} (1)
        (1) edge [bend left=15] node [near start, below=+1mm] {$b^2$} (0)
        (1) edge [loop above] node [right=+1mm] {$ab+ba$} ()
        (1) edge [bend left] node [above right=-1mm] {$a^2$} (2)
        (2) edge [bend left=15] node [near end, below left=-1.5mm] {$b^2$} (1)
        (2) edge [loop left] node {$ab+ba$} ()
        (2) edge [bend left] node [right] {$a^2$} (3)
        (3) edge [bend left=15] node [near start, above] {$b^2$} (2)
        (3) edge [loop below] node [right] {$ab+ba$} ()
        (3) edge [bend left] node [below left=-1.5mm] {$a^2$} (0)
        (0) edge [bend left=15] node [above right=-1.5mm] {$b^2$} (3)
    ;
\end{tikzpicture}}
        }
        \vspace{5mm}
        \centerline{
            \hspace{-15mm}%
            \subfloat[]{%
                \begin{tikzpicture}[my_automaton]
    \tikzstyle{every state}=[my_named_state]
    \tikzstyle{initial}=[my_initial_state]
    \tikzstyle{accepting}=[my_accepting_state]
    %\draw[help lines] (-1.8,1.8) grid (1.8,-1.8);
    \node[state, initial, accepting] (0) at (1.8,0) {0};
    \node[state] (2) at (1.8,-3.6) {2};
    \node[state] (3) at (0,-1.8) {3};

    \path
        (0) edge [loop above] node {$ab+ba+a^2{(ab+ba)}^*b^2$} ()
        (0) edge [bend left=30] node [right=-0.5mm] {$\re{X}_2$} (2)
        (2) edge [bend left=10] node [right=-0.5mm] {$\re{Y}_2$} (0)
        (2) edge [loop below] node {$ab+ba+b^2{(ab+ba)}^*a^2$} ()
        (2) edge [bend left] node [below left=-1mm] {$a^2$} (3)
        (3) edge [bend left] node [above=+0.7mm] {$b^2$} (2)
        (3) edge [loop left] node [below left=-2mm] {$ab+ba$} ()
        (3) edge [bend left] node [left=-0.5mm] {$a^2$} (0)
        (0) edge [bend left] node [above] {$b^2$} (3)
    ;
\end{tikzpicture}}%
            \quad
            \subfloat[Automaton~$\aut{A}_2$]{%
                \begin{tikzpicture}[my_automaton]
    \tikzstyle{every state}=[my_named_state]
    \tikzstyle{initial}=[my_initial_state]
    \tikzstyle{accepting}=[my_accepting_state]
    %\draw[help lines] (-1.8,1.8) grid (1.8,-1.8);
    \node[state, initial, accepting] (0) at (1.8,0) {0};
    \node[state] (1) at (1.8,-3.6) {1};

    \path
        (0) edge [loop above] node {$\re{Z}_2$} ()
        (0) edge [bend left] node [right] {$\re{X}_2 + \re{Y}_2$} (1)
        (1) edge [bend left] node [left] {$\re{X}_2 + \re{Y}_2$} (0)
        (1) edge [loop below] node {$\re{Z}_2$} ()
    ;
\end{tikzpicture}}
            \quad
            \subfloat[Automaton~$\aut{A}_3$]{%
                \begin{tikzpicture}[my_automaton]
    \tikzstyle{every state}=[my_named_state]
    \tikzstyle{initial}=[my_initial_state]
    \tikzstyle{accepting}=[my_accepting_state]
    %\draw[help lines] (-1.8,1.8) grid (1.8,-1.8);
    \node[state, initial, accepting] (0) at (1.8,0) {0};

    \path
        (0) edge [loop below] node {$\re{Z}_2 + {(\re{X}_2 + \re{Y}_2)}\re{Z}_2^*{(\re{X}_2 + \re{Y}_2)}$} ()
    ;
\end{tikzpicture}}
        }
        \caption{Steps of the state removal algorithm on $\aut{R}(8)$}\label{fig:automaton_R8_state_removal_steps}%
    \end{figure}
\end{example}

We have found a~rational expression of star height $q$ denoting the language $W_q$, which means that the star height of the language will not be higher then $q$. We proceed to show that it also has to be at least $q$. First let us present a few definitions and observe some of their properties that will be useful to us later.

\begin{defn}
    For any non-negative integers $k$ and $n$, \emph{witness word $w_{k,n}$} is a word defined recursively as:
    \begin{align*}
        w_{0,n} &= ab,\\
        w_{1,n} &= a^2 {(ab)}^n b^2 {(ab)}^n,\\
                &\; \vdots \\
        w_{k,n} &= a^{2^k} {(w_{k-1,n})}^n b^{2^k} {(w_{k-1,n})}^n.
    \end{align*}
\end{defn}

\begin{lemma}\label{lm:witness_words_equality}
    For any non-negative integers $k$ and $n$, $|w_{k,n}|_a = |w_{k,n}|_b$.
\end{lemma}

\begin{proof}
    By induction on $k$. For $k = 0$, $|ab|_a - |ab|_b = 0$ regardless of $n$. Given the induction hypothesis for some $k - 1$, $|w_{k-1,n}|_a - |w_{k-1,n}|_b = 0$. The word $w_{k,n}$ consists wholly of factors $a^{2^k}, \: b^{2^k}$, and two ${(w_{k-1,n})}^n$. Thus:
    \begin{multline*}
        |w_{k,n}|_a - |w_{k,n}|_b = |a^{2^k}|_a + 2|{(w_{k-1,n})}^n|_a + |b^{2^k}|_a - |a^{2^k}|_b - 2|{(w_{k-1,n})}^n|_b - |b^{2^k}|_b \\
        = 2^k + 2 \, n \, |w_{k-1,n}|_a + 2^k - 2^k - 2 \, n \, |w_{k-1,n}|_b - 2^k = 2 \, n \, ( \, |w_{k-1,n}|_a - |w_{k-1,n}|_b \, ) = 0.
    \end{multline*}
\end{proof}

From Lemma~\ref*{lm:witness_words_equality} it follows that any witness word $w_{k,n}$ is in any language $W_q$.

\begin{lemma}\label{lm:witness_words_inequalities}
    Any prefix $u$ and suffix $v$ of ${(w_{k,n})}^n$ satisfy the equations:
    \[
        0 \leq |u|_a - |u|_b \leq 2^{k+1}-1 \; , \qquad 0 \leq |v|_b - |v|_a \leq 2^{k+1}-1.
    \]
\end{lemma}

\begin{proof}
    First, we prove the inequalities of the prefix $u$. Let $u = {(w_{k,n})}^i u'$ such that $u'$ is a~proper prefix of $w_{k,n}$. Since
    \begin{multline*}
        |u|_a - |u|_b = |{(w_{k,n})}^i|_a + |u'|_a - |{(w_{k,n})}^i|_b - |u'|_b \\
        = i \, ( \, |w_{k,n}|_a - |w_{k,n}|_b \, ) + |u'|_a - |u'|_b = |u'|_a - |u'|_b,
    \end{multline*}
    we only need to prove that the inequalities hold for a prefix of $w_{k,n}$. This lets us assume, without loss of generality, that $u$ is proper prefix of $w_{k,n}$. We proceed by induction on $k$. For $k = 0$, $u$ is a~prefix of $ab$ and we can see that the assertion is true. Given the induction hypothesis for some $k - 1$, a prefix of $w_{k-1,n}$ has between $0$ and $2^k-1$ more $a$'s than $b$'s. The word
    \[
        w_{k,n} = a^{2^k} {(w_{k-1,n})}^n b^{2^k} {(w_{k-1,n})}^n
    \]
    begins with $2^k$ $a$'s, therefore if $u$ is a~prefix of $a^{2^k} {(w_{k-1,n})}^n$, $0 \leq |u|_a - |u|_b \leq 2^k+2^k-1 = 2^{k+1}-1$. That proves the inequalities of the prefix $u$.

    For the proof of inequalities of suffix $v$, we write $w_{k,n}$ as $w_{k,n} = uv$. Since $0 = |w_{k,n}|_b - |w_{k,n}|_a = |u|_b + |v|_b - |u|_a - |v|_a$,
    \[
        |v|_b - |v|_a = |u|_a - |u|_b.
    \]
    Therefore $0 \leq |v|_b - |v|_a \leq 2^{k+1}-1$.
\end{proof}

Next we use the witness words to define a property of languages.

\begin{defn}
    We say that language $L$ \emph{satisfies property $P_k$} if there exists an infinite number of values of $n$ such that ${(w_{k,n})}^n$ is a~factor of at~least one word in~$L$.
\end{defn}

Note that if $L$ satisfies $P_k$, it also satisfies $P_l$ for $l < k$ since ${(w_{l,n})}^n$ is a~factor of ${(w_{k,n})}^n$.

\begin{proof}[Proof of \autoref*{thm:main}]
    Lemma~\ref*{lm:expression_existence} shows that the star height of $W_q$ is at~most~$q$. Now we show that it also has to be at~least~$q$.

    By $\mathfrak{W}_k$ we denote a family of languages $L$ that satisfy the following conditions:
    \begin{itemize}
        \item[(i)] $L \subseteq W_q$,
        \item[(ii)] $L$ satisfies $P_k$,
        \item[(iii)] $L$ has a minimum star height of any language satisfying the first two conditions.
    \end{itemize}

    Let $h_k$ be the common value of the star height of the languages in $\mathfrak{W}_k$. Languages in $\mathfrak{W}_0$ are necessarily infinite, therefore $0 < h_0$. $P_k$ implies $P_{k-1}$, which means that $h_{k-1} \leq h_k$. $W_q$ obviously satisfies (i). It also satisfies $P_{q-1}$ since $|{(w_{q-1,n})}^n|_a = |{(w_{q-1,n})}^n|_b$ for each $n$. Because we have found a~rational expression of star height $q$ denoting $W_q$, it follows that $h_{q-1} \leq q$. Therefore we have
    \[
        0 < h_0 \leq h_1 \leq \dotsb \leq h_{q-1} \leq q.
    \]

    To prove the theorem it is enough to show that $h_{k-1} < h_k$ for each $k$, $k=1, \dotsc, q-1$. Let $L$ be in $\mathfrak{W}_k$. Due to Lemma~\ref*{lm:star_height_distributivity} $L$ can be written as a finite union of languages $E_j$, each denoted by a~rational expression of the form
    \[
        \re{E}_j = u_0{(\re{H}_1)}^*u_1 \dotsm u_{m-1}{(\re{H}_m)}^*u_m,
    \]
    where each rational expression $\re{H}_i$ denotes a~language~$H_i$ with star height less then or equal to $h_k-1$. Since $L$ satisfies $P_k$ and the union of the languages $E_j$ is finite, it follows that at least one of $E_j$ has to satisfy $P_k$. We can therefore safely assume that $L$ itself is denoted by a~rational expression of the same form as $E_j$.

    For each word $g$ in $H_i$, words $u_0 u_1 \dotsm u_m$ and $u_0 u_1 \dotsm u_{i-1} g u_i \dotsm u_m$ are both in language $L$. Therefore, because it has to be true that $|g|_a \equiv |g|_b \pmod{2^q}$, $H_i \subseteq W_q$.

    Since $L$ is of a minimal star height to satisfy $P_k$, none of the $H_i$ satisfies $P_k$. Now we need to show that some $H_i$ satisfies $P_{k-1}$. As a matter of fact, we will have $h_{k-1} \leq h_k - 1$.

    $L$ satisfies $P_k$, therefore words ${(w_{k,n})}^n$ are factors of~$L$ for arbitrarily large $n$. Lemma~\ref*{lm:block_star_lemma} shows that we can find $N$ large enough, such that there is infinitely many $n' \geq N$, that ${(w_{k,n'})}^{n'}$ is a factor of $L$, and, for each $n'$, we have infinitely many $l$'s, that ${(w_{k,n'})}^l$ is a factor of $L$.

    Let $H_i^*$ be the language recognised by $\re{H}_i^*$. Since $m$ is finite and fixed for $L$, there has to be infinitely many $n$'s, that ${(w_{k,n})}^n$ is a factor of $r_n \in H_i^*$ for some~$i$, $1 \leq i \leq m$. This means that $H_i^*$ satisfies $P_k$. Next we show that for these $n$'s, ${(w_{k-1,n})}^n$ is a factor of a word in $H_i$, meaning $H_i$ satisfies $P_{k-1}$.

    We write $r_n$ as a factorisation $r_n = g_0 g_1 \dotsm g_l$, where $g_j \in H_i$, for $0 \leq j \leq l$. If $w_{k,n}$, from ${(w_{k,n})}^n$, is a factor of some $g_j$, the condition is satisfied, since ${(w_{k-1,n})}^n$ is a factor of $w_{k,n}$. Otherwise, let us show how a~factor ${(w_{k,n})}^2$ is covered by the factorisation of $r_n$. Written explicitly, we have
    \[
        {(w_{k,n})}^2 = a^{2^k}{(w_{k-1,n})}^{n}b^{2^k}{(w_{k-1,n})}^{n}a^{2^k}{(w_{k-1,n})}^{n}b^{2^k}{(w_{k-1,n})}^{n}.
    \]
    Let us consider the $g_j$ that covers, at least partially, the factor $b^{2^k}$. There are two possibilities:
    \begin{itemize}
        \item[(i)] $b^{2^k}$ is a factor of $g_j$,
        \item[(ii)] a left factor of $b^{2^k}$ is a right factor of $g_j$.
    \end{itemize}

    In case (i), we have $g_j = v b^{2^k} u$. If $v$ covers the factor ${(w_{k-1,n})}^{n}$ to the left of $b^{2^k}$, or $u$ covers the factor ${(w_{k-1,n})}^{n}$ to the right of $b^{2^k}$, the condition is satisfied. If not, $u$ is a left factor of ${(w_{k-1,n})}^{n}$ and $v$ is a right factor. Set
    \[
        x = |g_j|_b - |g_j|_a, \quad y = |u|_a - |u|_b, \quad \text{ and } \quad z = |v|_b - |v|_a.
    \]
    Hence
    \[
        x = 2^k - (y - z).
    \]
    We see that $1 - 2^k \leq y - z \leq 2^k - 1$, from Lemma~\ref*{lm:witness_words_inequalities}. That gives us
    \[
        0 < x < 2^{k+1} \leq 2^q.
    \]
    which contradicts the fact that $g_j \in H_i \subseteq W_q$.

    In case (ii), we have $g_j = v b^{r}, \: 0 < r < 2^k$. If ${(w_{k-1,n})}^{n}$ is a factor of $v$, the condition is satisfied. Otherwise $v$ is a right factor of ${(w_{k-1,n})}^{n}$ and, similarly as above, we set
    \[
        x = |g_j|_b - |g_j|_a, \quad \text{ and } \quad z = |v|_b - |v|_a.
    \]
    Hence
    \[
        x = r + z.
    \]
    Therefore, due to Lemma~\ref*{lm:witness_words_inequalities},
    \[
        r \leq x \leq r + 2^k - 1 < 2^{k+1} \leq 2^q.
    \]
    which produces the same contradiction as the case (i).
\end{proof}